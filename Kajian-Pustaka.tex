\chapter{Kajian Pustaka}

\section{Time Seies method}
Menurut paper Kentang \cite{Kentang}, prsamaan SWE adalah
Berikut diberikan persamaan pengatur dari persamaan gelombang pada gitar

\begin{equation}\label{Pers1}
    a=b+U^{n+1}_{i+1}
\end{equation}
Persamaan (\ref{Pers1}) jadsbahdhavhdvah ajdbajdb
\begin{equation}\label{nama-rumus}
    \int_0^1 \frac{f(x)}{g(x)}\ {\rm dx}=\sin x
\end{equation}

\begin{equation}\label{nama-rumus1}
   \alpha \times \beta =\gamma^{3\alpha}
\end{equation}

\begin{figure}[h!]
    \centering
    \includegraphics[scale=0.1]{Tel-U-Logo.png}
    \caption{Caption}
    \label{fig:my_label1}
\end{figure} 

Rumus (\ref{nama-rumus}) merupakan \textit{contoh} persamaan matematika. persamaan matematika diatas diberi nama \textbackslash label\{nama-rumus\}. dengan $\alpha=\gamma \times 100$

\begin{figure}[h!]
    \centering
    \includegraphics[scale=0.3]{Tel-U-Logo.png}
    \caption{Caption}
    \label{fig:my_label}
\end{figure}

Lihat \textit{pada} Gambar \ref{fig:my_label}

\subsection{Cara memanggil pustaka}
Contoh pustaka prosiding \cite{doyen2014explicit}, jurnal \cite{gunawan2015hydrostatic} dan buku \cite{toro2013riemann}. Atau dapat juga mengguanakan dua pustaka atau lebih dalam \cite{gunawan2015hydrostatic,toro2013riemann}.
\chapter{Kajian Pustaka}

\section{Categorial Grammar}
Categorial Grammar (CG) merupakan sebuah istilah yang mencakup beberapa formalisme terkait yang diajukan
untuk sintaks dan semantik dari bahasa alami serta untuk bahasa logis dan matematis \cite{Steedman92catg}.
Karakteristik yang paling terlihat dari CG adalah bentuk esktrim dari leksikalismenya di mana beban utama
(atau bahkan seluruh beban) sintaksisnya ditanggung oleh leksikon.
Konstituen tata bahasa dalam \textit{categorial grammar} dan khususnya semua leksikal diasosiasikan
dengan suatu \textit{type} atau \say{\textit{category}} (dalam \textit{category theory}) yang
mendefinisikan potensi mereka untuk dikombinasikan dengan konstituen lain untuk menghasilkan konstituen
majemuk.
\textit{Category} tersebut adalah salah satu dari sejumlah kecil \textit{category} dasar (seperti NP)
atau \textit{functor} (dalam \textit{category theory}).

Ada beberapa notasi berbeda untuk \textit{category} dalam merepresentasikan \textit{directional}-nya.
Notasi yang paling umum digunakan adalah \say{\textit{slash notation}} yang dipelopori oleh Bar-Hilel,
Lambek, dan kemudian dimodifikasi dalam kelompok teori yang dibedakan sebagai tata bahasa
\say{\textit{combinatory}} \textit{categorial grammar} (CCG).
Sebagai contoh, \textit{category} $(S\backslash{}NP)/NP$ merupakan suatu \textit{functor} yang memiliki
dua buah notasi \textit{slash} yaitu $\backslash$ dan $/$.
Masing-masing notasi \textit{slash} tersebut merepresentasikan \textit{directionality} yang berbeda.
Notasi \textit{forward slash}, $/$, mengindikasikan bahwa argumen dari suatu \textit{functor}
$X/Y$ ada di bagian kanan atau dengan kata lain $Y$.
Adapun \textit{backward slash}, $\backslash$, mengindikasikan bahwa argumen dari suatu \textit{functor}
$X\backslash{}Y$ ada di bagian kiri atau dengan kata lain $X$.
Demikian itu, penggunaan notasi \textit{slash} yang tepat sangat penting dikarenakan hal ini dapat
mempengaruhi konstituen dari hasil \say{kombinasi} \textit{category}-nya.

\section{Combinatory Categorial Grammar}
Combinatory Categorial Grammar (CCG) merupakan salah satu formalisme tata bahasa yang gaya aturannya
diturunkan dari \textit{categorial grammar} dengan beberapa penambahan aturan dan istilah baru.
Di CCG, \textit{category} dapat dipasangkan dengan \textit{combinator}.
Dalam hal ini, \textit{combinator} yang dimaksud adalah abstraksi fungsi lambda
(dalam \textit{lambda calculus}).
Sebagai contoh, \textit{category} $(S\backslash{}NP)/NP$ dapat dipasangkan dengan fungsi lambda
$\lambda{x. fx}$ sehingga dapat ditulis menjadi $(S\backslash{}NP)/NP : \lambda{x. fx}$.
Adapun pemetaan dari suatu token kata ke \textit{category}-nya menggunakan notasi $\vdash$.
Sebagai contoh, anggap saja kita memiliki kamus pemetaan sebagai berikut.

\begin{small}
\begin{align*}
  {Pamungkas} &\vdash NP: \so{pamungkas}\\
  {Setyo} &\vdash NP: \so{setyo}\\
  {dan} &\vdash CONJ: \lambda x.\lambda y.\lambda f.\ (f\ x) \land (f\ y)\\
  {menyukai} &\vdash (S{\backslash}NP)/NP: \lambda x.\lambda y.\ suka(y, x)\\
  {rendang} &\vdash NP: \so{rendang}
\end{align*}
\end{small}

Dengan kamus seperti tersebut, apabila kita memiliki kalimat
\say{Pamungkas dan Setyo menyukai rendang}, maka akan kita dapatkan:

\begin{center}
  \bgroup
  \catcode`!=\active \def!{\upshape}
  \catcode`?=\active \def?#1{\makebox[0pt]{#1}}
  \catcode`^=\active \def^#1{\footnotesize{#1}}
  \catcode`*=\active \def*#1{\scriptsize{#1}}
  \tabbedShortstack{
    !^Pamungkas & & !^dan & & !^Setyo & & !^menyukai & & !^rendang &\\
    \TABcline{1,3,5,7,9}
    !^{$NP$} & &
      !^{$CONJ$} & &
      !^{$NP$} & &
      !^{$(S\backslash NP)/NP$} & &
      !^{$NP$} &\\
    !{*: \so{pamungkas}} & &
      !{*: $\lambda x.\lambda y.\lambda f.\ (f\ x) \land (f\ y)$} & &
      !{*: \so{setyo}} & &
      !{*: $\lambda x.\lambda y.\ suka(y, x)$} & &
      !{*: \so{rendang}} &
  }
  \egroup
\end{center}

Ada beberapa operasi yang dapat dilakukan dalam CCG. \textit{Operand} dari operasi
yang dimaksud adalah \textit{category}. Berdasarkan contoh di atas, akan ada tiga
operasi yang dijalankan yaitu \textit{coordination}, \textit{forward application},
dan \textit{backward application}.
Untuk mendapatkan hasil yang diinginkan, kita lakukan \textit{type rising} sebelum
\textit{backward application}.
Sehingga, kita dapatkan:

\begin{center}
  \bgroup
  \catcode`!=\active \def!{\upshape}
  \catcode`?=\active \def?#1{\makebox[0pt]{#1}}
  \catcode`^=\active \def^#1{\footnotesize{#1}}
  \catcode`*=\active \def*#1{\scriptsize{#1}}
  \tabbedLongunderstack{
    !^Pamungkas & & !^dan & & !^Setyo & & !^menyukai & & !^rendang &\\
    \TABcline{1,3,5,7,9}
    !^{$NP$} & &
      !^{$CONJ$} & &
      !^{$NP$} & &
      !^{$(S\backslash NP)/NP$} & &
      !^{$NP$} &\\
    !{*: \so{pamungkas}} & &
      !{*: $\lambda x.\lambda y.\lambda f.\ (f\ x) \land (f\ y)$} & &
      !{*: \so{setyo}} & &
      !{*: $\lambda x.\lambda y.\ suka(y, x)$} & &
      !{*: \so{rendang}} &\\
    \TABrule & \TABrule &
      \TABrule & \TABrule &
      \TABrule\CCGCOOR & &
      \TABrule & \TABrule &
      \TABrule\CCGFA &\\
    & &
      ?{^{$NP$}}
      \ \ \ \ \ \ \ \ \ 
      & & & &
      \ \ \ \ \ \ \ 
      ?{^{$S\backslash NP$}}
      & & &\\
    & &
      ?{*: $\lambda f.\ (f\ \so{pamungkas}) \land (f\ \so{setyo})$}
      \ \ \ \ \ \ \ \ \ 
      & & & &
      \ \ \ \ \ \ \ 
      ?{*: $\lambda y.\ suka(y, \so{rendang})$}
      & & &\\
    \TABrule & \TABrule &
      \TABrule & \TABrule &
      \TABrule\CCGTR & &
      & & &\\
    & &
      ?{^{$S/(S\backslash NP)$}}
      \ \ \ \ \ \ \ \ \ 
      & & & & & & &\\
    & &
      ?{*: $\lambda f.\ (f\ \so{pamungkas}) \land (f\ \so{setyo})$}
      \ \ \ \ \ \ \ \ \ 
      & & & & & & &\\
    \TABrule & \TABrule &
      \TABrule & \TABrule &
      \TABrule & \TABrule &
      \TABrule & \TABrule &
      \TABrule\CCGFA &\\
    & & &
      ?{^{$S$}}
      & & & & & &\\
    & & &
      ?{*: $suka(\so{pamungkas}, \so{rendang}) \land suka(\so{setyo}, \so{rendang})$}
      & & & & & &
  }
  \egroup
\end{center}

Berdasarkan hasil evaluasi tersebut, kita dapatkan \textit{query} \ref{ccg:query:1}
yang diperoleh dari kalimat \say{Pamungkas dan Setyo menyukai rendang}.
Demikian itu, komputer dapat melakukan komputasi berdasarkan \textit{query} yang telah diperoleh.

\begin{equation}\label{ccg:query:1}
  suka(\so{pamungkas}, \so{rendang}) \land suka(\so{setyo}, \so{rendang})
\end{equation}

Kegiatan tersebut merupakan apa yang disebut dengan CCG \textit{parsing}.
Untuk dapat melakukan parsing, CCG \textit{lexicon} diperlukan.
Untuk mendapatkan CCG \textit{lexicon} kita dapat menggunakan CCG \textit{supertagger}
yang akan melakukan pelabelan suatu token kata ke CCG \textit{lexicon} berdasarkan
pemetaannya.

\section{Category Theory}
\textit{Category Theory} (CT) merupakan formalisme yang dapat digunakan untuk memformalkan
struktur matematis.
CT mempelajari \textit{category} yang merupakan sebuah representasi dari suatu
abstraksi konsep matematis.
Suatu \textit{category} memiliki kumpulan \textit{object} dan \textit{morphism}.
Untuk mempermudah pemahaman mengenai CT, kita akan gunakan
\textit{category of set} (kategori dari himpunan) sebagai contoh.
Dalam \textit{category of set}, \textit{object}-nya adalah himpunan dan
\textit{morphism}-nya (terkadang disebut dengan \textit{arrow}) adalah fungsi
(\textit{function}, sebuah pemetaan).
Kemudian, pemetaan dari suatu \textit{category} $C$ ke \textit{category} $D$
yang dipetakan oleh $F$ ($F: C \rightarrow D$) disebut sebagai \textit{functor}.


\section{Lambda Calculus}
\textit{Lambda calculus} ({$\lambda$}\textit{-calculus}) merupakan sebuah formalisme yang dikembangkan
oleh Alonzo Church sebagai alat yang digunakan untuk memahami konsep komputasi yang efektif
\cite{DBLP:journals/corr/Rojas15}.
Formalisme {$\lambda$}\textit{-calculus} cukup populer dan bahkan dijadikan sebagai pondasi teori bagi
paradigma pemrograman \textit{functional programming}.
Konsep utama dari {$\lambda$}\textit{-calculus} adalah apa yang disebut dengan \textit{expression}.
Suatu \textit{expression} dalam {$\lambda$}\textit{-calculus} terdiri dari tiga bagian yaitu
\textit{lambda notation} ({$\lambda$}), \textit{argument} (seperti $a$, $b$, $c$, $x$, dan lain-lain),
dan \textit{body} yang dipisahkan dengan tanda titik.
Sebagai contoh, fungsi lambda ${\lambda}x. x$ merupakan sebuah fungsi identitas yang mengambil
argumen $x$ kemudian mengembalikan nilai $x$ itu sendiri.
Dalam hal ini, terlihat bahwa notasi {$\lambda$} merupakan sebuah penanda bagi suatu fungsi lambda.
Kemudian, pengubah $x$ setelah notasi {$\lambda$} merupakan argumen dari fungsi tersebut.
Selanjutnya, tanda titik merupakan pemisah antara \textit{head} dan \textit{body} fungsi lambda.
Terakhir, setelah tanda titik adalah \textit{body} dari suatu fungsi lambda yang mana berupa
\textit{expression}.

Untuk mempermudah pemahaman, {$\lambda$}\textit{-calculus} dapat diperlakukan seperti fungsi tanpa
nama. Sebagai contoh, fungsi lambda $({\lambda}x. x + 5)$ apabila diberikan nilai $2$ sehingga
menjadi $({\lambda}x. x + 5) 2$ akan dievaluasi menjadi ${\lambda}(2). (2) + 5$.
Demikian itu, nilai yang dikembalikan oleh fungsi tersebut adalah $7$.
Sama seperti fungsi pada umumnya, konsep ini bernama \textit{substition} (substitusi).
Memahami {$\lambda$}\textit{-calculus} dirasa perlu berhubung dalam tugas akhir ini
{$\lambda$}\textit{-calculus} digunakan sebagai bentuk formal di \textit{category}
dalam konteks CCG \textit{lexicon}. Meskipun {$\lambda$}\textit{-calculus} tidak sesederhana
yang dijelaskan sebelumnya, setidaknya memahami {$\lambda$}\textit{-calculus} seperti ini
sudah cukup untuk dapat membangun \textit{supertagger} yang ada di tugas akhir ini.


\section{Supertagging}
TBA.


\section{Maximum Entropy Model}
TBA.


% \section{Time Seies method}
% Menurut paper Kentang \cite{Kentang}, prsamaan SWE adalah
% Berikut diberikan persamaan pengatur dari persamaan gelombang pada gitar

% \begin{equation}\label{Pers1}
%     a=b+U^{n+1}_{i+1}
% \end{equation}
% Persamaan (\ref{Pers1}) jadsbahdhavhdvah ajdbajdb
% \begin{equation}\label{nama-rumus}
%     \int_0^1 \frac{f(x)}{g(x)}\ {\rm dx}=\sin x
% \end{equation}

% \begin{equation}\label{nama-rumus1}
%    \alpha \times \beta =\gamma^{3\alpha}
% \end{equation}

% \begin{figure}[h!]
%     \centering
%     \includegraphics[scale=0.1]{Tel-U-Logo.png}
%     \caption{Caption}
%     \label{fig:my_label1}
% \end{figure} 

% Rumus (\ref{nama-rumus}) merupakan \textit{contoh} persamaan matematika. persamaan matematika diatas diberi nama \textbackslash label\{nama-rumus\}. dengan $\alpha=\gamma \times 100$

% \begin{figure}[h!]
%     \centering
%     \includegraphics[scale=0.3]{Tel-U-Logo.png}
%     \caption{Caption}
%     \label{fig:my_label}
% \end{figure}

% Lihat \textit{pada} Gambar \ref{fig:my_label}

% \subsection{Cara memanggil pustaka}
% Contoh pustaka prosiding \cite{doyen2014explicit}, jurnal \cite{gunawan2015hydrostatic} dan buku \cite{toro2013riemann}. Atau dapat juga mengguanakan dua pustaka atau lebih dalam \cite{gunawan2015hydrostatic,toro2013riemann}.
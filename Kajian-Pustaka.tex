\chapter{Kajian Pustaka}

\section{Categorial Grammar}
Categorial Grammar (CG) merupakan sebuah istilah yang mencakup beberapa formalisme terkait yang diajukan
untuk sintaks dan semantik dari bahasa alami serta untuk bahasa logis dan matematis \cite{Steedman92catg}.
Karakteristik yang paling terlihat dari CG adalah bentuk esktrim dari leksikalismenya di mana beban utama
(atau bahkan seluruh beban) sintaksisnya ditanggung oleh leksikon.
Konstituen tata bahasa dalam \textit{categorial grammar} dan khususnya semua leksikal diasosiasikan
dengan suatu \textit{type} atau \say{\textit{category}} (dalam \textit{category theory}) yang
mendefinisikan potensi mereka untuk dikombinasikan dengan konstituen lain untuk menghasilkan konstituen
majemuk.
\textit{Category} tersebut adalah salah satu dari sejumlah kecil \textit{category} dasar (seperti NP)
atau \textit{functor} (dalam \textit{category theory}).

Ada beberapa notasi berbeda untuk \textit{category} dalam merepresentasikan \textit{directional}-nya.
Notasi yang paling umum digunakan adalah \say{\textit{slash notation}} yang dipelopori oleh Bar-Hilel,
Lambek, dan kemudian dimodifikasi dalam kelompok teori yang dibedakan sebagai tata bahasa
\say{\textit{combinatory}} \textit{categorial grammar} (CCG).
Sebagai contoh, \textit{category} $(S\backslash{}NP)/NP$ merupakan suatu \textit{functor} yang memiliki
dua buah notasi \textit{slash} yaitu $\backslash$ dan $/$.
Masing-masing notasi \textit{slash} tersebut merepresentasikan \textit{directionality} yang berbeda.
Notasi \textit{forward slash}, $/$, mengindikasikan bahwa argumen dari suatu \textit{functor}
$X/Y$ ada di bagian kanan atau dengan kata lain $Y$.
Adapun \textit{backward slash}, $\backslash$, mengindikasikan bahwa argumen dari suatu \textit{fucntor}
$X\backslash{}Y$ ada di bagian kiri atau dengan kata lain $X$.

\section{Combinatory Categorial Grammar}
TBA.


\section{Category Theory}
TBA.


\section{Lambda Calculus}
TBA.


\section{Supertagging}
TBA.


\section{Maximum Entropy Model}
TBA.


% \section{Time Seies method}
% Menurut paper Kentang \cite{Kentang}, prsamaan SWE adalah
% Berikut diberikan persamaan pengatur dari persamaan gelombang pada gitar

% \begin{equation}\label{Pers1}
%     a=b+U^{n+1}_{i+1}
% \end{equation}
% Persamaan (\ref{Pers1}) jadsbahdhavhdvah ajdbajdb
% \begin{equation}\label{nama-rumus}
%     \int_0^1 \frac{f(x)}{g(x)}\ {\rm dx}=\sin x
% \end{equation}

% \begin{equation}\label{nama-rumus1}
%    \alpha \times \beta =\gamma^{3\alpha}
% \end{equation}

% \begin{figure}[h!]
%     \centering
%     \includegraphics[scale=0.1]{Tel-U-Logo.png}
%     \caption{Caption}
%     \label{fig:my_label1}
% \end{figure} 

% Rumus (\ref{nama-rumus}) merupakan \textit{contoh} persamaan matematika. persamaan matematika diatas diberi nama \textbackslash label\{nama-rumus\}. dengan $\alpha=\gamma \times 100$

% \begin{figure}[h!]
%     \centering
%     \includegraphics[scale=0.3]{Tel-U-Logo.png}
%     \caption{Caption}
%     \label{fig:my_label}
% \end{figure}

% Lihat \textit{pada} Gambar \ref{fig:my_label}

% \subsection{Cara memanggil pustaka}
% Contoh pustaka prosiding \cite{doyen2014explicit}, jurnal \cite{gunawan2015hydrostatic} dan buku \cite{toro2013riemann}. Atau dapat juga mengguanakan dua pustaka atau lebih dalam \cite{gunawan2015hydrostatic,toro2013riemann}.
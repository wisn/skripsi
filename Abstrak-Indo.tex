\chapter*{Abstrak}

Riset pemrosesan bahasa natural untuk bahasa Indonesia saat ini terbilang sedikit.
Bahkan, masih banyak area riset yang belum tersentuh seperti contohnya
\textit{combinatory categorial grammar} (CCG).
CCG merupakan formalisme tatabahasa yang pada akhirnya dapat dimanfaatkan untuk memperoleh informasi
dari suatu kalimat.
Informasi tersebut diperoleh setelah melakukan \textit{parsing} berdasarkan formalisme CCG dengan
menggunakan CCG \textit{parser}.
Untuk dapat melakukan \textit{parsing}, CCG \textit{parser} membutuhkan CCG lexicon yang mengandung
bentuk formal dari suatu token kata.
Bentuk formal tersebut umumnya adalah \textit{combinatory logic}.
CCG \textit{lexicon} diperoleh dari proses pelabelan suatu token kata terhadap bentuk formalnya dengan
menggunakan \textit{supertagging}.
Proses \textit{supertagging} akan menghasilkan \textit{supertag} yang kemudian disebut sebagai CCG
\textit{lexicon} karena formalisme yang digunakan adalah formalisme CCG.

Tugas akhir dengan judul \textbf{\Title} berusaha untuk membangun versi awal dari CCG
\textit{supertagger} untuk bahasa Indonesia dengan harapan dapat menjadi inisiator riset
pemrosesan bahasa natural dengan tema CCG sehingga ke depannya akan ada lebih banyak riset
mengenai CCG yang tersedia.
\textit{Supertagger} tersebut akan dibangun dengan menggunakan model Maximum Entropy dan implementasinya
akan ditulis dalam bahasa pemrograman Haskell.


\vspace{0.5 cm}
\begin{flushleft}
{\textbf{Kata Kunci:}
  combinatory categorial grammar, supertagger, maximum entropy model, bahasa indonesia, haskell
}
\end{flushleft}

%% This document is created by 
%%  Dr. Putu Harry Gunawan
%% Edited by
%%  Wisnu Adi Nurcahyo

\documentclass[a4paper,12pt,oneside]{book}
\usepackage[utf8]{inputenc}
\usepackage{sectsty}
\usepackage{graphicx}
\usepackage{epstopdf}
\usepackage{algorithm}
\usepackage{algpseudocode}
\usepackage{array}
\usepackage[table]{xcolor}
\usepackage{anysize}
\usepackage{amsmath}
\usepackage{amssymb}
\usepackage[bahasa]{babel}
\usepackage{indentfirst} %Spasi untuk paragraf pertama
\usepackage{geometry}
\usepackage{multirow}% http://ctan.org/pkg/multirow
\usepackage{hhline}% http://ctan.org/pkg/hhline
\marginsize{4cm}{3cm}{3cm}{3cm} %{left}{right}{top}{bottom}
\usepackage[compact]{titlesec} 
\usepackage{etoolbox}
\usepackage{natbib} %for Harvard syle
\usepackage[bookmarks,hypertexnames=false,debug]{hyperref}
%\usepackage[pageanchor]{hyperref}
\usepackage{bookmark}
\usepackage{dirtytalk}
% \usepackage{ccg-latex}
% \usepackage{booktabs}

% CCG
\usepackage[TABcline]{tabstackengine}

\def\tbs{\,\textbackslash}
%

\makeatletter
\patchcmd{\ttlh@hang}{\parindent\z@}{\parindent\z@\leavevmode}{}{}
\patchcmd{\ttlh@hang}{\noindent}{}{}{}

% CCG
\newcommand{\so}[1]{\ensuremath{\mbox{\it #1}^\prime}}
\newcommand\CCGFA{%
  \rlap{\kern.5\dimexpr\tabbed@gap\relax\llap{\fboxsep=0pt\colorbox{white}{$>$}}}}
\newcommand\CCGBA{%
  \rlap{\kern.5\dimexpr\tabbed@gap\relax\llap{\fboxsep=0pt\colorbox{white}{$<$}}}}
\newcommand\CCGCOOR{%
  \rlap{\kern.5\dimexpr\tabbed@gap\relax\llap{\fboxsep=0pt\colorbox{white}{\footnotesize $<\&>$}}}}
\newcommand\CCGTR{%
  \rlap{\kern.5\dimexpr\tabbed@gap\relax\llap{\fboxsep=0pt\colorbox{white}{\footnotesize $>\text{T}$}}}}
%
\makeatother

% CCG
\setlength\TABruleshift{\dimexpr.5\ht\strutbox-.5\dp\strutbox}
\setstacktabbedgap{1ex}
\TABstackTextstyle{\itshape}
\setstackgap{S}{2pt}
%

\chapterfont{\centering}
\newcommand{\bigsize}{\fontsize{16pt}{14pt}\selectfont}
\chapterfont{\centering\bigsize\bfseries}
\sectionfont{\large\bfseries}
\usepackage{tikz}
\usetikzlibrary{shapes.geometric, arrows}
%\renewcommand{\chaptertitle}{BAB}
\renewcommand{\thechapter}{\Roman{chapter}}
\renewcommand\thesection{\arabic{chapter}.\arabic{section}}
\renewcommand\thesubsection{\thesection.\arabic{subsection}}
\renewcommand{\theequation}{\arabic{chapter}.\arabic{equation}}
\renewcommand{\thefigure}{\arabic{chapter}.\arabic{figure}}
\renewcommand{\thetable}{\arabic{chapter}.\arabic{table}}

\renewcommand\bibname{Daftar Pustaka}
\addto{\captionsbahasa}{\renewcommand{\bibname}{Daftar Pustaka}}
\usepackage{fancyhdr}
\pagestyle{fancy}
\lhead{}
\chead{}
\rhead{}
\lfoot{}
\cfoot{\thepage}
\rfoot{}
\renewcommand{\headrulewidth}{0pt}

\makeatletter

%%%%%%%%%%%%%%%%%%%%%%%%%%%%%%%%%%%%%%%%%%%%%%%%%%%%%%%%%%%%
%
%  Berikut adalah data-data yang wajib diisi oleh mahasiswa
%
%%%%%%%%%%%%%%%%%%%%%%%%%%%%%%%%%%%%%%%%%%%%%%%%%%%%%%%%%%%%

% Judul dalam bahasa Indonesia
\title{
  Membangun sebuah Combinatory Categorial Grammar (CCG) Supertagger
  Berbasis Maximum Entropy untuk Bahasa Indonesia
}\let\Title\@title

% Judul dalam bahasa Inggris
\newcommand{\EngTitle}{
  Building a Combinatory Categorial Grammar (CCG) Supertagger
  Based on the Maximum Entropy for Bahasa Indonesia
}

% Nama Penulis
\author{
  Wisnu Adi Nurcahyo
}\let\Author\@author

% NIM Penulis
\newcommand{\NIM}{1301160479}

% Prodi Penulis
\newcommand{\Prodi}{Informatika}

% Kelompok Keahlian TA
\newcommand{\KK}{Intelligence, Computing, and Multimedia (ICM)}

% Gelar
\newcommand{\Gelar}{Sains Komputasi}

% Tahun TA
\date{2019}\let\Date\@date

% Kelas TA
\newcommand{\Kelas}{NLP}

% Tanggal Pengesahan
\newcommand{\Tanggal}{13}

% Bulan Pengesahan
\newcommand{\Bulan}{November}

% Nama Pembimbing Satu
\newcommand{\PembimbingSatu}{Dr. Ade Romadhony, S.T., M.T.}

% NIP Pembimbing Satu
\newcommand{\NIPPembimbingSatu}{06840042}

% Nama Pembimbing Dua
\newcommand{\PembimbingDua}{Dr. Calon Pembimbing 2, M.Kom}

% NIP Pembimbing Dua
\newcommand{\NIPPembimbingDua}{123456}

% Nama Kaprodi
\newcommand{\Kaprodi}{Niken Dwi Wahyu Cahyani, S.T., M.Kom. PhD}

% NIP Kaprodi
\newcommand{\NIPKaprodi}{xxxxxxx}


\newif\ifPembimbingHanyaSatu

%%%% WARNING kode berikut ini diaktifkan Jika pembimbing hanya satu %%%%

\PembimbingHanyaSatutrue

%%%%%%%%%%%%%%%%%%%%%%%%%%%%%%%%%%%%%%%%%%%%%%%%%%%%%%%%%%%%%%%%%%

\makeatother
\linespread{1}


\begin{document}
%%%%%%%%%%%%%%%%%%%%%%%%%%%%%%%%%%%%%%%%%%%%%%%%%%%%%%%%%%
% Dimulai dengan cover, lembar persetujuan, Abstrak, dll
%%%%%%%%%%%%%%%%%%%%%%%%%%%%%%%%%%%%%%%%%%%%%%%%%%%%%%%%%%%
\pagenumbering{roman} 
\begin{titlepage}
\thispagestyle{empty}
%\vspace*{0.7cm}
\pdfbookmark{Cover}{ }
{\centering
\large
{\bigsize\bf \Title}\\
\vspace{ 2cm}
\rm
\textbf{Proposal Tugas Akhir}\\
\vspace{0.5 cm}
\textbf{Kelas TA \Kelas}\\
\vspace{0.5 cm}
\textbf{\Author}\\ \textbf{NIM: \NIM}\\ 

\vspace{1.5 cm}

\begin{figure}[h]
{\centering {\includegraphics[scale=0.17]{Tel-U-Logo}}\par}
\end{figure}

\vspace{2 cm}
{\bigsize\textbf{Program Studi Sarjana \Prodi}\\
\vspace{0.5 cm}
\textbf{Fakultas Informatika}\\
\vspace{0.5 cm}
\textbf{Universitas Telkom}\\
\vspace{0.5 cm}
\textbf{Bandung}\\
\vspace{0.5 cm}
\textbf{\Date}\\}
}
\pagebreak
\thispagestyle{empty}
\pdfbookmark{Lembar Persetujuan}{ }
{\centering
\textbf{\large Lembar Persetujuan}\\
\vspace{0.5cm}
\textbf{\Title}\\
\vspace{0.5cm}
\textbf{\textit{\EngTitle}}\\
\vspace{0.5cm}
\textbf{\Author}\\
\textbf{NIM: \NIM}\\
\vspace{1cm}


{ Proposal ini diajukan sebagai usulan pembuatan tugas akhir pada\\ Program Studi Sarjana \Prodi\\ Fakultas Informatika Universitas Telkom}\\

\vspace{0.5cm}

{Bandung, \Tanggal\quad \Bulan \quad \Date}\\
{Menyetujui}\\

\vspace{0.5cm}
      \ifPembimbingHanyaSatu
             \begin{center}
            Calon Pembimbing 1
            \end{center}
       \else
          \begin{center}
              \begin{tabular}{  m{8cm}  m{8cm} }
              Calon Pembimbing 1 & Calon Pembimbing 2
              \end{tabular}
           \end{center}
      \fi
\begin{center}
\vspace{2cm}
\ifPembimbingHanyaSatu
     \underline{\PembimbingSatu} \\ 
      NIP: \NIPPembimbingSatu
\else
\begin{tabular}{  m{8cm}  m{8cm} }
\underline{\PembimbingSatu} & \underline{\PembimbingDua} \\ 
NIP: \NIPPembimbingSatu & NIP: \NIPPembimbingDua
\end{tabular}
\fi
\end{center}
\vspace{0.5cm}

}
\pagebreak
\end{titlepage}
\phantomsection
\addcontentsline{toc}{chapter}{Abstrak}
%\pdfbookmark{\abstractname}{Abstrak Indonesia}
\chapter*{Abstrak}

Dalam pemrosesan bahasa alami, combinatory categorial grammar (CCG) merupakan salah satu
formalisme tata bahasa yang dapat digunakan untuk membangun sebuah \textit{parser} yang umumnya
dikenal sebagai CCG \textit{parser}.
CCG \textit{parser} dapat digunakan untuk berbagai macam keperluan dalam pemrosesan bahasa alami.
Sebagai contoh, CCG \textit{parser} dapat digunakan untuk memperoleh informasi
(\textit{information extraction}) dari suatu kalimat yang kemudian membentuk sebuah \textit{query}.
Agar dapat membangun CCG \textit{parser} ataupun \textit{tools} lainnya, \textit{dataset} CCG
yang memuat \textit{lexicon} dibutuhkan. Saat Tugas Akhir ini ditulis \textit{dataset} CCG untuk
bahasa Indonesia belum tersedia. Demikian itu, CCGtown dibangun sebagai \textit{annotation tool}
yang dapat digunakan untuk membangun \textit{dataset} CCG.
CCGtown merupakan \textit{open source graphical annotation tool} berbasis web yang dirancang
dengan fokus mempercepat serta mempermudah proses penganotasian.
Dengan fitur \textit{generate} CCG \textit{derivation} serta \textit{auto-assign} CCG
\textit{category} kegiatan repetitif dalam melakukan anotasi akhirnya berkurang.
Dengan antarmuka (UI/UX) yang cukup baik menjadikan CCGtown alat anotasi yang mudah digunakan
(\textit{user friendly}) oleh berbagai kalangan pengguna.

\vspace{0.5 cm}
\begin{flushleft}
{\textbf{Kata Kunci:}
  natural language processing, combinatory categorial grammar, annotation tool
}
\end{flushleft}

\cleardoublepage
\phantomsection
\addcontentsline{toc}{chapter}{Daftar Isi}
%\pdfbookmark{\contentsname}{Contents}
\tableofcontents

%%%%%%%%%%%%%%%%%%%%%%%%%%%%%%%%%%%%%%%%%%%%%%%%%
% Bab bab penulisan
%%%%%%%%%%%%%%%%%%%%%%%%%%%%%%%%%%%%%%%%%%%%%%%%
\cleardoublepage
\pagenumbering{arabic}
\chapter{Pendahuluan}
\section{Latar Belakang}

Riset pemrosesan bahasa alami untuk bahasa Indonesia saat ini masih terbilang sedikit.
Bahkan, masih banyak area riset yang belum tersentuh seperti contohnya
\textit{combinatory categorial grammar} (CCG).
Sementara itu, riset mengenai CCG untuk bahasa Inggris sudah cukup matang.
Adapun untuk bahasa lainnya (seperti bahasa Vietnam) sudah mulai menggunakan CCG di dalam
penelitiannya \cite{nguyen2019vietnamese}.
Agar dapat menerapkan CCG di dalam aplikasi yang dibangun, \textit{tools} seperti
CCG \textit{parser} dan CCG \textit{supertagger} harus tersedia terlebih dahulu.
Masing-masing dari \textit{tools} tersebut memerlukan \textit{dataset} agar dapat memberikan
hasil yang akurat.

Umumnya terdapat dua cara yang paling sering digunakan untuk mengembangkan CCG \textit{supertagger}
maupun CCG \textit{parser} bahasa lokal yaitu (1) membangun \textit{dataset} CCG \textit{supertag}
secara manual maupun semi-otomatis atau (2) melakukan transfer \textit{dataset} dari CCGbank
(atau dari sumber lainnya) ke dalam bahasa lokal dengan cara melakukan alih bahasa dan bila perlu
melakukan penyesuaian untuk \textit{supertag}-nya \cite{hockenmaier-steedman-2007-ccgbank}.
Proses pembangunan \textit{dataset} umumnya menggunakan bantuan \textit{annotation tool} agar
proses anotasinya menjadi lebih mudah.
Salah satu \textit{annotation tool} yang dapat digunakan adalah
CCGweb \cite{evang-etal-2019-ccgweb}.

Tugas akhir dengan judul \say{\Title} berusaha untuk membangun alat anotasi CCG baru dengan
UI/UX yang lebih baik dari CCGweb.
Selain itu, dengan bantuan NLTK alat anotasi ini dapat melakukan \textit{generate} untuk
CCG \textit{derivation}-nya kemudian pengguna juga dapat mengubah \textit{derivation}-nya
apabila diperlukan.
Tujuan dari dibangunnya alat anotasi CCG ini adalah untuk mempermudah proses anotasi yang
repetitif.
Selanjutnya, \textit{dataset} CCG pertama untuk bahasa Indonesia diharapkan dapat dipublikasikan.
\pagebreak

\section{Perumusan Masalah}
Rumusan masalah yang akan diangkat yaitu:
\begin{enumerate}
    \item Bagaimana proses pembangunan alat anotasi untuk CCG?
    \item Apakah CCGtown dapat digunakan dengan mudah?
\end{enumerate}
\section{Tujuan}
Tujuan yang diharapkan dapat tercapai oleh tugas akhir ini yaitu:
\begin{enumerate}
    \item Membangun dan merilis alat anotasi CCG semi-otomatis.
    \item Dapat digunakan untuk membangun \textit{dataset} CCG.
\end{enumerate}
\section{Rencana Kegiatan}
Rencana kegiatan yang akan dilakukan adalah sebagai berikut:
\begin{itemize}
    \item Studi literatur
    \item Studi \textit{tools} yang tersedia
    \item Studi bahasa pemrograman yang akan digunakan
    \item Perancangan sistem \textit{annotation tool} CCGtown
    \item Membangun \textit{annotation tool} CCGtown
    \item Menguji kemudahan dalam menggunakan CCGtown
\end{itemize}

% I don't know why but this table is not showing up
\section{Jadwal Kegiatan}
Laporan proposal ini akan dijadwalkan sesuai dengan tabel \ref{ScheduleTable}.

\begin{table}
  \centering
    \caption{Jadwal kegiatan proposal tugas akhir.}
  \label{ScheduleTable}
  \begin{tabular}{|c|m{2.5cm}|m{0.01cm}|m{0.01cm}|m{0.01cm}|m{0.01cm}|m{0.01cm}|m{0.01cm}|m{0.01cm}|m{0.01cm}|m{0.01cm}|m{0.01cm}|m{0.01cm}|m{0.01cm}|m{0.01cm}|m{0.01cm}|m{0.01cm}|m{0.01cm}|m{0.01cm}|m{0.01cm}|m{0.01cm}|m{0.01cm}|m{0.01cm}|m{0.01cm}|m{0.01cm}|m{0.01cm}|}
    \hline
    \multirow{2}{*}{\textbf{No}} & \multirow{2}{*}{\textbf{Kegiatan}} & \multicolumn{24}{|c|}{\textbf{Bulan ke-}} \\
    \hhline{~~------------------------}
    {} & {} & \multicolumn{4}{|c|}{\textbf{1}} & \multicolumn{4}{|c|}{\textbf{2}} & \multicolumn{4}{|c|}{\textbf{3}} & \multicolumn{4}{|c|}{\textbf{4}} & \multicolumn{4}{|c|}{\textbf{5}} & \multicolumn{4}{|c|}{\textbf{6}}\\
    \hline
    1 & Studi Literatur & \cellcolor{blue!25} & \cellcolor{blue!25} & \cellcolor{blue!25} & \cellcolor{blue!25}& \cellcolor{blue!25} & \cellcolor{blue!25} & \cellcolor{blue!25} & \cellcolor{blue!25}& \cellcolor{blue!25} & \cellcolor{blue!25} & \cellcolor{blue!25} & \cellcolor{blue!25}& \cellcolor{blue!25} & \cellcolor{blue!25} & \cellcolor{blue!25} & \cellcolor{blue!25}& \cellcolor{blue!25} & \cellcolor{blue!25} & \cellcolor{blue!25} & \cellcolor{blue!25}& \cellcolor{blue!25} & \cellcolor{blue!25} & \cellcolor{blue!25} & \cellcolor{blue!25}\\
    \hline
    2 & Studi \textit{Tools} yang Tersedia & \cellcolor{blue!25} & \cellcolor{blue!25} & \cellcolor{blue!25} & \cellcolor{blue!25} &  \cellcolor{blue!25} &  \cellcolor{blue!25} &  \cellcolor{blue!25} &  \cellcolor{blue!25} & {} & {} & {} & {}& {} & {} & {} & {}& {} & {} & {} & {}& {} & {} & {} & {}\\
    \hline
    3 & Studi Bahasa Pemrograman dan Framework-nya & {} & {} & {} & {} & \cellcolor{blue!25} & \cellcolor{blue!25} & \cellcolor{blue!25} & \cellcolor{blue!25} & {} & {} & {} & {}& {} & {} & {} & {}& {} & {} & {} & {}& {} & {} & {} & {}\\
    % \hline
    % 4 & Pengumpulan Data & {} & {} & {} & {} & \cellcolor{blue!25} & \cellcolor{blue!25} & \cellcolor{blue!25} & \cellcolor{blue!25} & \cellcolor{blue!25} & \cellcolor{blue!25} & \cellcolor{blue!25} & \cellcolor{blue!25} & {} & {} & {} & {}& {} & {} & {} & {} & {} & {} & {} & {}\\
    \hline
    5 & Analisis dan Perancangan Sistem &  {} & {} & {} & {}  & \cellcolor{blue!25} & \cellcolor{blue!25} & \cellcolor{blue!25} & \cellcolor{blue!25} & \cellcolor{blue!25} & \cellcolor{blue!25} & \cellcolor{blue!25} & \cellcolor{blue!25} & {} & {} & {} & {}& {} & {} & {} & {}& {} & {} & {} & {}\\
    \hline
    6 & Implementasi Sistem &  {} & {} & {} & {} & {} & {} & {} & {}& \cellcolor{blue!25} & \cellcolor{blue!25} & \cellcolor{blue!25} & \cellcolor{blue!25} & \cellcolor{blue!25} & \cellcolor{blue!25} & \cellcolor{blue!25} & \cellcolor{blue!25} & {} & {} & {} & {}& {} & {} & {} & {}\\
    \hline
    7 & Analisa Hasil Implementasi &  {} & {} & {} & {} & {} & {} & {} & {}& {} & {} & {} & {} & \cellcolor{blue!25} & \cellcolor{blue!25} & \cellcolor{blue!25} & \cellcolor{blue!25} & \cellcolor{blue!25} & \cellcolor{blue!25} & \cellcolor{blue!25} & \cellcolor{blue!25} & {} & {} & {} & {}\\
    \hline
    8 & Penulisan Laporan & {} & {} & {} & {} & \cellcolor{blue!25} & \cellcolor{blue!25} & \cellcolor{blue!25} & \cellcolor{blue!25}& \cellcolor{blue!25} & \cellcolor{blue!25} & \cellcolor{blue!25} & \cellcolor{blue!25}& \cellcolor{blue!25} & \cellcolor{blue!25} & \cellcolor{blue!25} & \cellcolor{blue!25}& \cellcolor{blue!25} & \cellcolor{blue!25} & \cellcolor{blue!25} & \cellcolor{blue!25}& \cellcolor{blue!25} & \cellcolor{blue!25} & \cellcolor{blue!25} & \cellcolor{blue!25}\\
    \hline
  \end{tabular}
\end{table}

%
\chapter{Kajian Pustaka}

\section{Categorial Grammar}
Categorial Grammar (CG) merupakan sebuah istilah yang mencakup beberapa formalisme terkait yang diajukan
untuk sintaks dan semantik dari bahasa alami serta untuk bahasa logis dan matematis \cite{Steedman92catg}.
Karakteristik yang paling terlihat dari CG adalah bentuk esktrim dari leksikalismenya di mana beban utama
(atau bahkan seluruh beban) sintaksisnya ditanggung oleh leksikon.
Konstituen tata bahasa dalam \textit{categorial grammar} dan khususnya semua leksikal diasosiasikan
dengan suatu \textit{type} atau \say{\textit{category}} (dalam \textit{category theory}) yang
mendefinisikan potensi mereka untuk dikombinasikan dengan konstituen lain untuk menghasilkan konstituen
majemuk.
\textit{Category} tersebut adalah salah satu dari sejumlah kecil \textit{category} dasar (seperti NP)
atau \textit{functor} (dalam \textit{category theory}).

Ada beberapa notasi berbeda untuk \textit{category} dalam merepresentasikan \textit{directional}-nya.
Notasi yang paling umum digunakan adalah \say{\textit{slash notation}} yang dipelopori oleh Bar-Hilel,
Lambek, dan kemudian dimodifikasi dalam kelompok teori yang dibedakan sebagai tata bahasa
\say{\textit{combinatory}} \textit{categorial grammar} (CCG).
Sebagai contoh, \textit{category} $(S\backslash{}NP)/NP$ merupakan suatu \textit{functor} yang memiliki
dua buah notasi \textit{slash} yaitu $\backslash$ dan $/$.
Masing-masing notasi \textit{slash} tersebut merepresentasikan \textit{directionality} yang berbeda.
Notasi \textit{forward slash}, $/$, mengindikasikan bahwa argumen dari suatu \textit{functor}
$X/Y$ ada di bagian kanan atau dengan kata lain $Y$.
Adapun \textit{backward slash}, $\backslash$, mengindikasikan bahwa argumen dari suatu \textit{fucntor}
$X\backslash{}Y$ ada di bagian kiri atau dengan kata lain $X$.

\section{Combinatory Categorial Grammar}
TBA.


\section{Category Theory}
TBA.


\section{Lambda Calculus}
TBA.


\section{Supertagging}
TBA.


\section{Maximum Entropy Model}
TBA.


% \section{Time Seies method}
% Menurut paper Kentang \cite{Kentang}, prsamaan SWE adalah
% Berikut diberikan persamaan pengatur dari persamaan gelombang pada gitar

% \begin{equation}\label{Pers1}
%     a=b+U^{n+1}_{i+1}
% \end{equation}
% Persamaan (\ref{Pers1}) jadsbahdhavhdvah ajdbajdb
% \begin{equation}\label{nama-rumus}
%     \int_0^1 \frac{f(x)}{g(x)}\ {\rm dx}=\sin x
% \end{equation}

% \begin{equation}\label{nama-rumus1}
%    \alpha \times \beta =\gamma^{3\alpha}
% \end{equation}

% \begin{figure}[h!]
%     \centering
%     \includegraphics[scale=0.1]{Tel-U-Logo.png}
%     \caption{Caption}
%     \label{fig:my_label1}
% \end{figure} 

% Rumus (\ref{nama-rumus}) merupakan \textit{contoh} persamaan matematika. persamaan matematika diatas diberi nama \textbackslash label\{nama-rumus\}. dengan $\alpha=\gamma \times 100$

% \begin{figure}[h!]
%     \centering
%     \includegraphics[scale=0.3]{Tel-U-Logo.png}
%     \caption{Caption}
%     \label{fig:my_label}
% \end{figure}

% Lihat \textit{pada} Gambar \ref{fig:my_label}

% \subsection{Cara memanggil pustaka}
% Contoh pustaka prosiding \cite{doyen2014explicit}, jurnal \cite{gunawan2015hydrostatic} dan buku \cite{toro2013riemann}. Atau dapat juga mengguanakan dua pustaka atau lebih dalam \cite{gunawan2015hydrostatic,toro2013riemann}.
%
% \chapter{Perancangan Sistem}
CCGtown dibangun dengan menggunakan bahasa pemrograman Python dan JavaScript.
Adapun \textit{framework} yang digunakan adalah Django.
Versi awal CCGtown merupakan sebuah \textit{proof-of-concept} dari
\textit{open source graphical annotation tool} berbasis web yang dilengkapi dengan fitur
penganotasian semi-otomatis.
Bahasa pemrograman Python digunakan karena sebagian besar \textit{library} untuk CCG
sudah tersedia di PyPi \footnote{\url{https://pypi.org/}}.
Salah satu \textit{library} penting yang digunakan sebagai dasar dari fitur penganotasian
semi-otomatis adalah NTLK \footnote{\url{http://www.nltk.org/}}.
Selanjutnya, Django digunakan untuk mempercepat proses pengembangan aplikasi.
Adapun JavaScript digunakan untuk menjadikan CCGtown aplikasi berbasis web yang interaktif.

Alur kerja CCGtown pada umumnya adalah (1) pengguna melakukan registrasi, (2) pengguna
melakukan \textit{login} ke sistem, (3) pengguna membuat proyek baru, (4) pengguna
menambahkan kalimat yang ingin dianotasi, (5) pengguna melakukan anotasi kemudian melakukan
\textit{generate} CCG \textit{derivation} dan/atau melakukan modifikasi
\textit{derivation}-nya apabila diperlukan, dan (6) pengguna melakukan \textit{export}
setelah selesai melakukan anotasi. Alur kerja tersebut mempengaruhi desain sistem dari
CCGtown. Salah satunya adalah desain dari \textit{database} yang akan digunakan.


\section{Desain Database}
CCGtown menggunakan PostgreSQL sebagai
DBMS\footnote{Database Management System}-nya.
Hal ini karena PostgreSQL memiliki kemampuan untuk menyimpan struktur data
JSON\footnote{JavaScript Object Notation} sehingga memudahkan CCGtown untuk menyimpan
format JSON dari CCG \textit{derivation} yang telah dimanipulasi oleh pengguna melalui
fitur \textit{editable CCG derivation}.
PostgreSQL juga memiliki banyak fitur lain termasuk di antaranya dukungan
dari \textit{non-relational database model} (seperti \textit{multi-model graph})
sehingga apabila di waktu yang akan datang CCGtown memerlukan perubahan signifkan
terhadap desain \textit{database}-nya tidak perlu mengganti DBMS yang digunakan.
Fitur lain seperti \textit{function} dan \textit{procedure} juga akan sangat membantu
pengembangan CCGtown di waktu yang akan datang.

CCGtown versi awal sejatinya hanya membutuhkan tiga tabel saja yaitu tabel
\textit{accounts} untuk menyimpan pengguna yang terdaftar, tabel
\textit{projects} untuk menyimpan proyek-proyek yang sudah dibuat, dan tabel
\textit{sentences} untuk menyimpan kalimat-kalimat yang akan dianotasikan.
Tiga tabel tersebut sudah cukup untuk membangun \textit{proof-of-concept} dari
alat anotasi CCG yang akan dibangun. Adapun
ERD\footnote{Entity Relationship Diagram}-nya dapat dilihat pada Gambar\ref{erd-1}.

\begin{figure}\centering\small
  \scalebox{.75}{
  \begin{tikzpicture}[node distance=1.5cm, every edge/.style={link}]
    \node[entity] (acc) {Accounts};
    \node[attribute] (acc-id) [above=of acc] {\key{ID}} edge (acc);
    \node[attribute] (acc-uuid) [above right=of acc] {\key{UUID}} edge (acc);
    \node[attribute] (acc-email) [right=of acc] {Email} edge (acc);
    \node[attribute] (acc-password) [below right=of acc] {Password} edge (acc);

    \node[relationship] (creates) [left=of acc] {Creates} edge (acc);

    \node[entity] (prj) [below=of creates] {Projects} edge (creates);
    \node[attribute] (prj-id) [above right=of prj] {\key{ID}} edge (prj);
    \node[attribute] (prj-uuid) [above left=of prj] {\key{UUID}} edge (prj);
    \node[attribute] (prj-author) [left=of prj] {Author ID} edge (prj);
    \node[attribute] (prj-name) [below left=of prj] {Name} edge (prj);
    \node[attribute] (prj-status) [below=of prj] {Status} edge (prj);
    \node[attribute] (prj-rules) [below right=of prj] {Rules} edge (prj);

    \node[relationship] (adds) [right=0.5cm and 2cm of prj] {Adds} edge (prj);

    \node[entity] (snt) [below=of adds] {Sentences} edge (adds);
    \node[attribute] (snt-id) [left=of snt] {\key{ID}} edge (snt);
    \node[attribute] (snt-uuid) [below left=of snt] {\key{UUID}} edge (snt);
    \node[attribute] (snt-project) [below=of snt] {Project ID} edge (snt);
    \node[attribute] (snt-words) [below right=of snt] {Words} edge (snt);
    \node[attribute] (snt-cats) [right=of snt] {Categories} edge (snt);
    \node[attribute] (snt-deriv) [above right=of snt] {Derivations} edge (snt);
  \end{tikzpicture}
  }
  \caption{Conceptual Entity Relationship Diagram (ERD) CCGtown}
  \label{erd-1}
\end{figure}

Masing-masing tabel memiliki dua \textit{key} yaitu $ID$ dan
$UUID$\footnote{Universally Unique IDentifier}.
$ID$ merupakan \textit{primary key} \textit{integer} dengan \textit{auto increment}
yang berfungsi sebagai \textit{identifier} untuk melakukan operasi
\textit{update} maupun \textit{delete}.
Adapun $UUID$ merupakan \textit{indexed column} yang berfungsi sebagai
\textit{indentifier} publik (dapat dilihat oleh pengguna melalui URL)
yang mana digunakan untuk operasi \textit{read}.
$ID$ tidak digunakan sebagai \textit{identifier} publik karena pengguna dapat
melakukan \textit{brute-force} untuk mencari proyek ataupun kalimat berdasarkan
$ID$ yang bukan miliknya.
Demikian itu alasan ditambahkannya atribut $UUID$.
Alasan kenapa CCGtown tetap menyimpan kolom $ID$ adalah karena $ID$ nantinya akan
digunakan untuk membuat \textit{pagination}.

Pada tabel $accounts$, selain $ID$ dan $UUID$ juga memiliki atribut $email$ dan
$password$. Masing-masing atribut tersebut menggunakan tipe data \textit{string}
atau VARCHAR di PostgreSQL.
Tabel $accounts$ memiliki hubungan \textit{one-to-many} terhadap tabel $projects$.
Adapun atribut tabel $projects$ adalah $author\_id$, $name$, $status$, dan $rules$.
Atribut $author\_id$ merupakan \textit{foreign key} (\textit{indexed}) yang
mengarah kepada tabel $accounts$ dan tipe data yang digunakan sama dengan
atribut $ID$ yang terdapat di tabel $accounts$.
Atribut $name$ menggunakan tipe data \textit{string} (VARCHAR).
Atribut $status$ menggunakan tipe data \textit{integer} yang berperan sebagai
\textit{enum} ($0$ = \textit{just created}, $1$ = \textit{in progress},
$2$ = \textit{finished}, dan $3$ = \textit{dropped}).
Tabel $projects$ memiliki hubungan \textit{one-to-many} terhadap tabel $sentences$.
Adapun atribut tabel $sentences$ adalah $project\_id$, $words$, $categories$, dan
$derivations$. Atribut $project\_id$ merupakan \textit{foreign key}
(\textit{indexed}) yang mengarah kepada tabel $projects$ dan tipe data yang digunakan
sama dengan atribut $ID$ yang terdapat di tabel $projects$.
Sisanya, atribut $words$, $categories$, dan $derivations$ menggunakan tipe data JSON.


\section{Desain Sistem}
CCGtown sejatinya memiliki desain sistem yang cukup sederhana.
Fungsionalitas yang akan didukung untuk versi awal adalah (1) \textit{register} dan
\textit{login}, (2) manajemen proyek (CRUD\footnote{Create, Read, Update, Delete}
), (3) dan manajemen kalimat (CRUD).
Pada manajemen kalimat, CCGtown menggunakan JavaScript untuk membuat pembuatan
maupun perubahan CCG \textit{derivation} menjadi lebih interaktif.
Selain tiga fungsionalitas tersebut, CCGtown juga menambahkan fungsionalitas tambahan
seperti \textit{auto-assign category} yang dilakukan di sisi \textit{frontend}.
Kemudian, CCGtown juga menambahkan fungsionalitas tambahan di sisi \textit{backend}
yaitu CCG \textit{derivation generator} dengan memanfaatkan \textit{library} NLTK
dan kemampuan untuk melakukan \textit{export} CCG \textit{derivation} yang disimpan
di \textit{database}.

Pengguna harus terdaftar terlebih dahulu sebelum dapat melaukan anotasi sehingga
langkah awal yang harus dibangun adalah fungsionalitas \textit{register}.
Alur proses pendaftaran pengguna dapat dilihat pada Gambar \ref{flowchart:register}.
Berhubung fokus saat ini adalah \textit{proof-of-concept}, informasi yang dibutuhkan
untuk mendaftar hanyalah \textit{email} dan \textit{password}. Adapun
\textit{password confirmation} digunakan untuk memvalidasi \textit{password}
sehingga dapat mengurangi risiko pengguna melupakan
\textit{password}-nya yang baru saja di-\textit{input}.
Saat pengguna melakukan pendaftaran, sistem akan memeriksa apakah \textit{email} yang
didaftar sudah terdapat di \textit{database}.
Apabila sudah terdaftar, pengguna akan dialihkan ke halaman \textit{register} kembali
dan mendapatkan \textit{flash message} dengan keterangan "email sudah terdaftar".
Sebaliknya, sistem akan melakukan \textit{input} data tersebut ke dalam \textit{database}
lalu mengalihkan pengguna ke halaman \textit{login}.
Ketika dialihkan ke halaman \textit{login}, pengguna akan melihat \textit{flash message}
dengan keterangan "pengguna berhasil didaftarkan".
Pada tahap ini pengguna sudah dapat melakukan \textit{login} ke dalam sistem CCGtown.

\begin{figure}\centering\small
  \scalebox{.75}{
	\begin{tikzpicture}[node distance=2cm]
    \node (start) [cloud] {Start};
    \node (input) [io, below of=start, yshift=0.5cm] {\textit{Input email, password,} dan \textit{password confirmation}};
    \node (check) [process, below of=input, yshift=0.5cm] {Mencari pengguna berdasarkan \textit{email}};
    \node (is-registered) [decision, below of=check, yshift=-1.25cm] {\textit{Email} sudah terdaftar?};
    \node (registered) [process, right of=is-registered, yshift=0cm, xshift=4.5cm] {\textit{Redirect} ke halaman \textit{register}};
    \node (registering) [process, below of=is-registered, yshift=-1.75cm] {\textit{Input} informasi pengguna ke \textit{database}};
    \node (redirect) [process, below of=registering, yshift=0.5cm] {\textit{Redirect} ke halaman \textit{login}};
    \node (stop) [cloud, below of=redirect, yshift=0.5cm] {Stop};

    \draw [arrow] (start) -- (input);
    \draw [arrow] (input) -- (check);
    \draw [arrow] (check) -- (is-registered);
    \draw [arrow] (is-registered) -- node[anchor=south]{Ya} (registered);
    \draw [arrow] (registered) -- +(3,0) |- (input);
    \draw [arrow] (is-registered) -- node[anchor=east]{Tidak} (registering);
    \draw [arrow] (registering) -- (redirect);
    \draw [arrow] (redirect) -- (stop);
  \end{tikzpicture}
  }
	\caption{Alur proses pendaftaran pengguna.}
  \label{flowchart:register}
\end{figure}

Pada proses "\textit{input} informasi pengguna ke \textit{database}" CCGtown melakukan
\textit{password hashing} dengan menggunakan Bcrypt.
Informasi sensitif seperti \textit{password} sebaiknya tidak disimpan sebagai
\textit{plain text}. Demikian itu CCGtown menggunakan \textit{password hashing}.
Apabila hal buruk terjadi seperti misalnya \textit{data breach} (kebocoran data),
\textit{password} pengguna tidak dapat langsung digunakan.
Peretas perlu mencari cara untuk memecahkan \textit{password} tersebut.
Bcrypt merupakan skema \textit{password hashing} berbasis Blowfish \textit{block cipher}
yang didesain untuk lebih \textit{resistant} terhadap serangan \textit{brute-force}
\cite{bcrypt}.
Serangan \textit{brute-force} merupakan upaya peretas untuk menebak \textit{password}
dengan cara membuat \textit{wordlist} yang kemudian dicocokkan dengan \textit{hash}
yang terbentuk satu-demi-satu.
Meskipun terjadi \textit{data breach}, peretas perlu usaha ekstra untuk dapat menebak
\textit{password} dari satu pengguna.
Hal ini mengurangi kerugian yang akan dialami oleh CCGtown apabila \textit{data breach}
benar-benar terjadi.

Selanjutnya, setelah melakukan registrasi, pengguna dapat melakukan \textit{login} ke
sistem CCGtown.
Proses yang dilakukan pada umumnya sama dengan aplikasi web yang memiliki
kemampuan \textit{register} dan \textit{login}. Alur proses \textit{login} dapat dilihat
pada Gambar \ref{flowchart:login}.
Setelah pengguna melakukan \textit{input} \textit{email} dan \textit{password}-nya,
CCGtown akan melakukan pencarian di \textit{database} apakah \textit{email} yang
diberikan terdaftar. Apabila tidak terdaftar, pengguna akan dialihkan ke halaman
\textit{login} dan diberikan \textit{flash message} "\textit{Email} dan/atau
\textit{password} tidak cocok". Pesan ini diberikan agar peretas tidak dapat mencari
tahu \textit{email} mana saja yang sudah terdaftar. Selanjutnya, apabila akun
dengan \textit{email} tersebut ada, maka langkah selanjutnya adalah mencocokkan
\textit{password} yang diberikan oleh pengguna dan \textit{password} yang telah
disimpan di \textit{database}. Kemudian, sistem melakukan Bcrypt \textit{sync}.
Apabila tidak berhasil, pengguna akan dialihkan ke halaman \textit{login}
dan diberikan \textit{flash message} "\textit{Email} dan/atau \textit{password}
tidak cocok". Sebaliknya, pengguna akan dialihkan ke halaman Projects yang berisi
daftar proyek yang telah dibuat sebelumnya.

\begin{figure}\centering\small
  \scalebox{.75}{
	\begin{tikzpicture}[node distance=2cm]
    \node (start) [cloud] {Start};
    \node (input) [io, below of=start, yshift=0.5cm] {\textit{Input email} dan \textit{password}};
    \node (check) [process, below of=input, yshift=0.5cm] {Mencari pengguna berdasarkan \textit{email}};
    \node (is-registered) [decision, below of=check, yshift=-1.25cm] {\textit{Email} sudah terdaftar?};
    \node (not-registered) [process, left of=is-registered, yshift=0cm, xshift=-4.5cm] {\textit{Redirect} ke halaman \textit{login}};
    \node (logging-in) [process, below of=is-registered, yshift=-1.75cm] {Mencocokkan \textit{password} dengan Bcrypt \textit{sync}};
    \node (is-matched) [decision, below of=logging-in, yshift=-1.25cm] {\textit{Password} cocok?};
    \node (redirect) [process, below of=is-matched, yshift=-1.75cm] {\textit{Redirect} ke halaman Projects};
    \node (stop) [cloud, below of=redirect, yshift=0.5cm] {Stop};

    \draw [arrow] (start) -- (input);
    \draw [arrow] (input) -- (check);
    \draw [arrow] (check) -- (is-registered);
    \draw [arrow] (is-registered) -- node[anchor=south]{Tidak} (not-registered);
    \draw [arrow] (not-registered.north) -- +(0,0) |- (input);
    \draw [arrow] (is-registered) -- node[anchor=east]{Ya} (logging-in);
    \draw [arrow] (logging-in) -- (is-matched);
    \draw [arrow] (is-matched.west) -| +(-4.6,0) -- node[anchor=east]{Tidak} (not-registered.south);
    \draw [arrow] (is-matched) -- node[anchor=east]{Ya} (redirect);
    \draw [arrow] (redirect) -- (stop);
  \end{tikzpicture}
  }
	\caption{Alur proses \textit{login} ke sistem CCG.}
  \label{flowchart:login}
\end{figure}

Pada halaman Projects, pengguna dapat membuat proyek atau menghapus proyek.
Tidak ada fungsionalitas spesial di halaman Projects selain CRUD pada umumnya.
Satu pengguna dapat membuat banyak proyek. Tidak ada larangan tertentu terhadap penamaan
proyek. Namun, sangat disarankan memberikan nama proyek yang deskriptif seperti misalnya
"\textit{Wide-range Indonesian Dataset}". Setiap proyek memiliki status yang berbeda-beda.
Proyek yang baru saja dibuat akan memiliki status \textit{just created}.
Hal ini untuk memudahkan \textit{annotator} mencari proyek mana yang baru akan dikerjakan,
proyek mana yang sedang dikerjakan, proyek mana yang sudah selesai dikerjakan, atau
proyek mana yang tidak jadi dikerjakan. Proyek yang telah dibuat dapat disunting maupun
dihapus. Proyek yang dihapus tidak dapat dikembalikan (\textit{undo}).
Adapun penyuntingan proyek terjadi di halaman Editor.

Pada halaman Editor, pengguna dapat menyunting informasi proyek seperti nama proyek,
status proyek, dan \textit{rules} yang akan digunakan untuk melakukan \textit{generate}
CCG \textit{derivation} via NTLK. Selain itu, pengguna juga dapat menambahkan kalimat
baru yang akan dianotasi. Pengguna dapat menambahkan lebih dari satu kalimat sekaligus.
Kalimat-kalimat tersebut akan di-\textit{tokenize} menggunakan \textit{library} NLTK.
Ekstensi yang digunakan untuk proses \textit{tokenize} ini adalah punkt.
Setelah itu, barulah pengguna dapat melakukan penganotasian terhadap kalimat-kalimat yang
telah ditambahkan. Terdapat dua cara untuk memberikan anotasi yaitu secara langsung di
halaman Editor atau dapat juga dilakukan di Editable CCG Modal.
Saat ini CCGtown belum mendukung penganotasian terhadap \textit{compound words}.
CCGtown saat ini juga belum mendukung penganotasian CCG dengan semantik.
Versi awal CCGtown hanya mendukung penganotasian CCG secara sintaksis saja.

Setelah semua kata dalam suatu kalimat diberikan anotasi, pengguna dapat melakukan
\textit{generate} CCG \textit{derivation}. Hal ini dapat dilakukan berkat bantuan
\textit{library} NLTK. Kami mengambil sebuah $rules$ dari tabel $projects$ dan kemudian
kami mengambil semua $words$ serta $categories$ dari tabel $sentences$ yang merupakan
bagian dari proyek tersebut. Kolom $words$ merupakan kumpulan kata dari kalimat yang telah
di-\textit{tokenize}. Adapun kolom $categories$ merupakan anotasi CCG \textit{category}-nya.
\textit{Pseudocode} untuk \textit{generate} CCG \textit{derivation} dapat dilihat pada
Kode \ref{code:ccg-gen} dengan asumsi anotasi yang diberikan absah (dapat dibuat CCG
\textit{derivation}-nya). Kode $next$ tersebut akan mengambil satu dari banyak
kemungkinan \textit{derivation} yang dapat dibuat. Contoh \textit{object} yang
di-\textit{return} dapat dilihat pada Kode \ref{code:ccg-gen-example}.
Untuk kepentingan \textit{rendering} di sisi \textit{frontend}, \textit{key} seperti
$from$ dan $to$ sangat diperlukan. \textit{Key} $from$ dan \textit{key} $to$
merepresentasikan \textit{index} posisi terhadap \textit{array} $words$.
Dengan bantuan kedua \textit{key} tersebut, \textit{frontend} dapat melakukan kalkulasi
posisi masing-masing elemen yang terdapat di \textit{object} $derivations$.

\begin{lstlisting}[
  language=python,
  firstnumber=1,
  caption={Pseudocode untuk melakukan \textit{generate} CCG \textit{derivation}.},
  label={code:ccg-gen}
]
from nltk.ccg import chart, lexicon

def generateCCGDerivation(rules, words, categories, target_words):
    lex = rules + '\n\n'
    for i in range(len(words)):
        lex += words[i] + ' => ' + categories[i] + '\n'

    lex = lexicon.parseLexicon(lex)
    parser = chart.CCGChartParser(lex, chart.DefaultRuleSet)
    result = next(parser.parse(target_words))
    derivations = makeCCGDeriv(result)

    return derivations
\end{lstlisting}

Kode \ref{code:ccg-gen-example} didapatkan dari fungsi $makeCCGDeriv$ yang terdapat pada
Kode \ref{code:ccg-gen}. Fungsi $makeCCGDeriv$ sederhananya mengambil $Tree$ yang didapatkan
dari $parser.parse$ kemudian melakukan \textit{tree traversal}. Semua \textit{leaf}, diambil
dari paling "kiri", diletakkan di elemen pertama $derivations$. Selanjutnya, kita berjalan
melalui \textit{parent} dari \textit{leaf} tersebut hingga ke \textit{root} mencari bentuk CCG
\textit{derivation}-nya. Banyaknya baris yang dibutuhkan oleh CCG \textit{derivation} dapat
dilihat dari \textit{height} yang dimiliki oleh $Tree$ tersebut. Kemudian, hasil dari CCG
\textit{derivation} (umumnya berupa $S$) merupakan elemen terakhir $derivations$.

\begin{lstlisting}[
  language=json,
  firstnumber=1,
  caption={Contoh \textit{derivations object} yang di-\textit{return}.},
  label={code:ccg-gen-example}
]
[
  [
    { "to": 0, "from": 0, "word": "You" },
    { "to": 1, "from": 1, "word": "prefer" },
    { "to": 2, "from": 2, "word": "that" },
    { "to": 3, "from": 3, "word": "cake" }
  ],
  [
    { "to": 0, "from": 0, "category": "NP" },
    { "to": 1, "from": 1, "category": "((S\NP)/NP)" },
    { "to": 2, "from": 2, "category": "(NP/N)" },
    { "to": 3, "from": 3, "category": "N" }
  ],
  [
    { "to": 3, "from": 2, "category": "NP", "operator": ">" }
  ],
  [
    { "to": 3, "from": 1, "category": "(S\NP)", "operator": ">" }
  ],
  [
    { "to": 3, "from": 0, "category": "S", "operator": "<" }
  ]
]
\end{lstlisting}


\section{Pertimbangan UI/UX}
TBA.

%

%%%%%%%%%%%%%%%%%%%%%%%%%%%%%%%%%%%%%%%%%%%%%%%%%%%%%%%%%%%%%%%
% Berikut adalah untuk membuat Daftar pustaka
% Style daftar pustaka dapat disesuaikan dengan mengubah 
%\bibliographystyle{kode}
% dengan kode = acm maka hasinya contoh [1]
% dengan kode = agsm maka hasilnya Harvard style (Sukimin, 2017)
%%%%%%%%%%%%%%%%%%%%%%%%%%%%%%%%%%%%%%%%%%%%%%%%%%%%%%%%%%%%%%%%
\cleardoublepage
\addcontentsline{toc}{chapter}{Daftar Pustaka}
\bibliographystyle{plain} %harvard style
\bibliography{References}
%
%\pagebreak
%%%%%%%%%%%%%%%%%%%%%%%%%%%%%%%%%%%%%%%%%%%%%%%%%%%%%%%%%%%%%%%
% Berikut untuk lampiran
%%%%%%%%%%%%%%%%%%%%%%%%%%%%%%%%%%%%%%%%%%%%%%%%%%%%%%%%%%%%%%
\cleardoublepage
\addcontentsline{toc}{chapter}{Lampiran}
\include{Lampiran}
\end{document}

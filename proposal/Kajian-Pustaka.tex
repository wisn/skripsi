\chapter{Kajian Pustaka}

\section{Categorial Grammar}
Categorial Grammar (CG) merupakan sebuah istilah yang mencakup beberapa formalisme terkait yang diajukan
untuk sintaks dan semantik dari bahasa alami serta untuk bahasa logis dan matematis \cite{Steedman92catg}.
Karakteristik yang paling terlihat dari CG adalah bentuk ekstrim dari leksikalismenya di mana beban utama
(atau bahkan seluruh beban) sintaksisnya ditanggung oleh leksikon.
Konstituen tata bahasa dalam \textit{categorial grammar} dan khususnya semua leksikal diasosiasikan
dengan suatu \textit{type} atau \say{\textit{category}} (dalam \textit{category theory}) yang
mendefinisikan potensi mereka untuk dikombinasikan dengan konstituen lain untuk menghasilkan konstituen
majemuk.
\textit{Category} tersebut adalah salah satu dari sejumlah kecil \textit{category} dasar (seperti NP)
atau \textit{functor} (dalam \textit{category theory}).
Dalam hal ini, \textit{category} dapat diartikan sebagai \textit{syntactic type} dari suatu kata.

Secara formal, \textit{syntactic type} didefinisikan sebagai himpunan bagian dari suatu
\textit{semigroup} $M$ yang tunduk pada tiga operasi yaitu \ref{catg:syn:1},
\ref{catg:syn:2}, dan \ref{catg:syn:3} dimana $A$, $B$, dan $C$ merupakan himpunan bagian dari $M$
\cite{Lambek1988}. Adapun $A \cdot B$ dibaca $A$ \textit{times} $B$, $C/B$ dibaca $C$ \textit{over}
$B$, dan $A\backslash{}C$ dibaca $A$ \textit{under} $C$. Selanjutnya, dapat dilihat bahwasannya
untuk semua $A, B, C \subseteq M$ sehingga kita dapatkan \ref{catg:syn:4} dan \ref{catg:syn:5}.
Terakhir, persamaan \ref{catg:syn:6} dapat diabaikan apabila dihadapkan dengan
\textit{multiplicative system} yang tidak asosiatif. Sementara itu, apabila \textit{semigroup}-nya
merupakan sebuah \textit{monoid} dengan identitas $1$ maka kita dapatkan \ref{catg:syn:7} dimana
$I = \{1\}$.

\begin{align}
  \begin{split}\label{catg:syn:1}
    A \cdot B & = \{x \cdot y \in M \mid x \in A \land y \in B\}
  \end{split}\\
  \begin{split}\label{catg:syn:2}
    C/B & = \{x \in M \mid \forall_{y \in B} x \cdot y \in C\}
  \end{split}\\
  \begin{split}\label{catg:syn:3}
    A\backslash{}C & = \{y \in M \mid \forall_{x \in A} x \cdot y \in C\}
  \end{split}
\end{align}

\begin{align}
  \begin{split}\label{catg:syn:4}
    A \cdot B \subseteq C & \;\;\;\;\text{jika dan hanya jika}\;\;\;\; A \subseteq C/B
  \end{split}\\
  \begin{split}\label{catg:syn:5}
    A \cdot B \subseteq C & \;\;\;\;\text{jika dan hanya jika}\;\;\;\; B \subseteq A\backslash{}C
  \end{split}
\end{align}

\begin{align}
  \begin{split}\label{catg:syn:6}
    (A \cdot B) \cdot C = A \cdot (B \cdot C)
  \end{split}\\
  \begin{split}\label{catg:syn:7}
    I \cdot A = A = A \cdot I
  \end{split}
\end{align}

Ada beberapa notasi berbeda untuk \textit{category} dalam merepresentasikan \textit{directional}-nya.
Notasi yang paling umum digunakan adalah \say{\textit{slash notation}} yang dipelopori oleh Bar-Hilel,
Lambek, dan kemudian dimodifikasi dalam kelompok teori yang dibedakan sebagai tata bahasa
\say{\textit{combinatory}} \textit{categorial grammar} (CCG).
Sebagai contoh, \textit{category} $\text{(S$\backslash$NP)/NP}$ merupakan suatu \textit{functor} yang
memiliki dua buah notasi \textit{slash} yaitu $\backslash$ dan $/$.
Masing-masing notasi \textit{slash} tersebut merepresentasikan \textit{directionality} yang berbeda.
Notasi \textit{forward slash}, $/$, mengindikasikan bahwa argumen dari suatu \textit{functor}
$\text{X}/\text{Y}$ ada di bagian kanan atau dengan kata lain $\text{Y}$.
Adapun \textit{backward slash}, $\backslash$, mengindikasikan bahwa argumen dari suatu \textit{functor}
$\text{X}\backslash\text{Y}$ ada di bagian kiri atau dengan kata lain $\text{X}$.
Demikian itu, penggunaan notasi \textit{slash} yang tepat sangat penting dikarenakan hal ini dapat
mempengaruhi konstituen dari hasil \say{kombinasi} \textit{category}-nya.


\section{Combinatory Categorial Grammar}\label{kajian-ccg}
Combinatory Categorial Grammar (CCG) merupakan salah satu formalisme tata bahasa yang gaya aturannya
diturunkan dari \textit{categorial grammar} dengan beberapa penambahan aturan dan istilah baru
\cite{Steedman96avery}.
Di CCG, \textit{category} dapat dipasangkan dengan \textit{semantic representation}.
Dalam hal ini, \textit{semantic representation} yang dimaksud adalah abstraksi fungsi lambda
(dalam \textit{lambda calculus}, \textit{lambda function}).
Sebagai contoh, \textit{category} $\text{(S$\backslash$NP)/NP}$ dapat dipasangkan dengan fungsi lambda
$\lambda{x. fx}$ sehingga dapat ditulis menjadi $\text{(S$\backslash$NP)/NP} : \lambda{x. fx}$.
Adapun pemetaan dari suatu token kata ke \textit{category}-nya menggunakan notasi $\vdash$.
Sebagai contoh, anggap saja kita memiliki kamus pemetaan seperti pada Gambar \ref{ccg:mapping:1}.
Apabila kita memiliki kalimat \say{Pamungkas dan Setyo menyukai rendang}, maka kita dapatkan:

\begin{figure}\centering\small
  \begin{align*}
    \text{Pamungkas} &\ \vdash\ \text{NP}:\ \so{pamungkas}\\
    \text{Setyo} &\ \vdash\ \text{NP}:\ \so{setyo}\\
    \text{dan} &\ \vdash\ \text{CONJ}:\ \lambda x.\lambda y.\lambda f.\ (f\ x) \land (f\ y)\\
    \text{menyukai} &\ \vdash\ \text{(S{$\backslash$}NP)/NP}:\ \lambda x.\lambda y.\ suka(y, x)\\
    \text{rendang} &\ \vdash\ \text{NP}:\ \so{rendang}
  \end{align*}
  \caption{Kamus yang memetakan token kata ke bentuk CCG \textit{lexicon}-nya.}
  \label{ccg:mapping:1}
\end{figure}

\begin{center}
  \bgroup
  \catcode`!=\active \def!{\upshape}
  \catcode`?=\active \def?#1{\makebox[0pt]{#1}}
  \catcode`^=\active \def^#1{\footnotesize{#1}}
  \catcode`*=\active \def*#1{\scriptsize{#1}}
  \tabbedShortstack{
    !^Pamungkas & & !^dan & & !^Setyo & & !^menyukai & & !^rendang &\\
    \TABcline{1,3,5,7,9}
    !^{$\text{NP}$} & &
      !^{$\text{CONJ}$} & &
      !^{$\text{NP}$} & &
      !^{$\text{(S$\backslash$NP)/NP}$} & &
      !^{$\text{NP}$} &\\
    !{*: \so{pamungkas}} & &
      !{*: $\lambda x.\lambda y.\lambda f.\ (f\ x) \land (f\ y)$} & &
      !{*: \so{setyo}} & &
      !{*: $\lambda x.\lambda y.\ suka(y, x)$} & &
      !{*: \so{rendang}} &
  }
  \egroup
\end{center}

Ada beberapa operasi yang dapat dilakukan dalam CCG. \textit{Operand} dari operasi
yang dimaksud adalah \textit{category}. Berdasarkan contoh di atas, akan ada tiga
operasi yang dijalankan yaitu \textit{coordination}, \textit{forward application},
dan \textit{type rising}.
Untuk mendapatkan hasil yang diinginkan, kita lakukan \textit{type rising} sebelum
\textit{forward application} di akhir.
Sehingga, kita dapatkan:

\begin{center}
  \bgroup
  \catcode`!=\active \def!{\upshape}
  \catcode`?=\active \def?#1{\makebox[0pt]{#1}}
  \catcode`^=\active \def^#1{\footnotesize{#1}}
  \catcode`*=\active \def*#1{\scriptsize{#1}}
  \tabbedLongunderstack{
    !^Pamungkas & & !^dan & & !^Setyo & & !^menyukai & & !^rendang &\\
    \TABcline{1,3,5,7,9}
    !^{$\text{NP}$} & &
      !^{$\text{CONJ}$} & &
      !^{$\text{NP}$} & &
      !^{$\text{(S$\backslash$NP)/NP}$} & &
      !^{$\text{NP}$} &\\
    !{*: \so{pamungkas}} & &
      !{*: $\lambda x.\lambda y.\lambda f.\ (f\ x) \land (f\ y)$} & &
      !{*: \so{setyo}} & &
      !{*: $\lambda x.\lambda y.\ suka(y, x)$} & &
      !{*: \so{rendang}} &\\
    \TABrule & \TABrule &
      \TABrule & \TABrule &
      \TABrule\CCGCOOR & &
      \TABrule & \TABrule &
      \TABrule\CCGFA &\\
    & &
      !?{^{$\text{NP}$}}
      \ \ \ \ \ \ \ \ \ 
      & & & &
      \ \ \ \ \ \ \ 
      !?{^{$\text{S$\backslash$NP}$}}
      & & &\\
    & &
      ?{*: $\lambda f.\ (f\ \so{pamungkas}) \land (f\ \so{setyo})$}
      \ \ \ \ \ \ \ \ \ 
      & & & &
      \ \ \ \ \ \ \ 
      ?{*: $\lambda y.\ suka(y, \so{rendang})$}
      & & &\\
    \TABrule & \TABrule &
      \TABrule & \TABrule &
      \TABrule\CCGTR & &
      & & &\\
    & &
      !?{^{$\text{S/(S$\backslash$NP)}$}}
      \ \ \ \ \ \ \ \ \ 
      & & & & & & &\\
    & &
      ?{*: $\lambda f.\ (f\ \so{pamungkas}) \land (f\ \so{setyo})$}
      \ \ \ \ \ \ \ \ \ 
      & & & & & & &\\
    \TABrule & \TABrule &
      \TABrule & \TABrule &
      \TABrule & \TABrule &
      \TABrule & \TABrule &
      \TABrule\CCGFA &\\
    & & &
      !?{^{$\text{S}$}}
      & & & & & &\\
    & & &
      ?{*: $suka(\so{pamungkas}, \so{rendang}) \land suka(\so{setyo}, \so{rendang})$}
      & & & & & &
  }
  \egroup
\end{center}

Berdasarkan hasil evaluasi tersebut, kita dapatkan \textit{query} \ref{ccg:query:1}
yang diperoleh dari kalimat \say{Pamungkas dan Setyo menyukai rendang}.
Demikian itu, komputer dapat melakukan komputasi berdasarkan \textit{query} yang telah diperoleh.
Kegiatan tersebut merupakan apa yang disebut dengan CCG \textit{parsing}.
Untuk dapat melakukan parsing, CCG \textit{lexicon} diperlukan.
Untuk mendapatkan CCG \textit{lexicon} kita dapat menggunakan CCG \textit{supertagger}
yang akan melakukan pelabelan suatu token kata ke CCG \textit{lexicon} berdasarkan
pemetaannya.

\begin{equation}\label{ccg:query:1}
  suka(\so{pamungkas}, \so{rendang}) \land suka(\so{setyo}, \so{rendang})
\end{equation}


\section{Lambda Calculus}
\textit{Lambda calculus} ({$\lambda$}\textit{-calculus}) merupakan sebuah formalisme yang dikembangkan
oleh Alonzo Church sebagai alat yang digunakan untuk memahami konsep komputasi yang efektif
\cite{DBLP:journals/corr/Rojas15}.
Formalisme {$\lambda$}\textit{-calculus} cukup populer dan bahkan dijadikan sebagai pondasi teori bagi
paradigma pemrograman \textit{functional programming}.
Konsep utama dari {$\lambda$}\textit{-calculus} adalah apa yang disebut dengan \textit{expression}.
Suatu \textit{expression} dalam {$\lambda$}\textit{-calculus} terdiri dari tiga bagian yaitu
\textit{lambda notation} ({$\lambda$}), \textit{argument} (seperti $a$, $b$, $c$, $x$, dan lain-lain),
dan \textit{body} yang dipisahkan dengan tanda titik.
Sebagai contoh, fungsi lambda ${\lambda}x. x$ merupakan sebuah fungsi identitas yang mengambil
argumen $x$ kemudian mengembalikan nilai $x$ itu sendiri.
Dalam hal ini, terlihat bahwa notasi {$\lambda$} merupakan sebuah penanda bagi suatu fungsi lambda.
Kemudian, pengubah $x$ setelah notasi {$\lambda$} merupakan argumen dari fungsi tersebut.
Selanjutnya, tanda titik merupakan pemisah antara \textit{head} dan \textit{body} fungsi lambda.
Terakhir, setelah tanda titik adalah \textit{body} dari suatu fungsi lambda yang mana berupa
\textit{expression}.

Untuk mempermudah pemahaman, {$\lambda$}\textit{-calculus} dapat diperlakukan seperti fungsi tanpa
nama. Sebagai contoh, fungsi lambda $({\lambda}x. x + 5)$ apabila diberikan nilai $2$ sehingga
menjadi $({\lambda}x. x + 5) 2$ akan dievaluasi menjadi ${\lambda}(2). (2) + 5$.
Demikian itu, nilai yang dikembalikan oleh fungsi tersebut adalah $7$.
Sama seperti fungsi pada umumnya, konsep ini bernama \textit{substition} (substitusi).
Memahami {$\lambda$}\textit{-calculus} dirasa perlu berhubung dalam tugas akhir ini
{$\lambda$}\textit{-calculus} digunakan sebagai bentuk formal di \textit{category}
dalam konteks CCG \textit{lexicon}. Meskipun {$\lambda$}\textit{-calculus} tidak sesederhana
yang dijelaskan sebelumnya, setidaknya memahami {$\lambda$}\textit{-calculus} seperti ini
sudah cukup untuk dapat membangun \textit{supertagger} yang ada di tugas akhir ini.


\section{CCGweb}
CCGweb\footnote{\url{https://github.com/texttheater/ccgweb}} merupakan
\textit{open source graphical annotation tool} pertama untuk CCG \cite{evang-etal-2019-ccgweb}.
Aplikasinya berbasis web dan dibangun dengan menggunakan bahasa pemrograman
Python, PHP, dan JavaScript.
Fitur yang paling menarik dari \textit{graphical annotation tool} adalah What You See Is What
You Get (WYSIWYG) yang mana berupa kemampuan untuk me-\textit{render} CCG \textit{derivation}
sesuai dengan apa yang kita lihat.
Maksudnya, CCG \textit{derivation} akan ditampilkan horizontal sesuai dengan panjang kalimatnya
kemudian hasil \textit{derivation}-nya ditampilkan vertikal seperti contoh pada
Bagian \ref{kajian-ccg}.

Untuk dapat menggunakan CCGweb, kita perlu melakukan instalasi terlebih dahulu.
Selanjutnya barulah kita dapat menambahkan kalimat-kalimat yang ingin dianotasi.
Satu instalasi CCGweb hanya dapat digunakan untuk satu proyek anotasi sehingga
apabila kita memiliki lebih dari satu proyek maka kita perlu melakukan instalasi
CCGweb yang baru.
Demikian itu, CCGtown\footnote{\url{https://github.com/wisn/ccgtown}} hadir dengan
fitur \textit{multi-project} dan tanpa perlu melakukan instalasi di komputer lokal
karena aplikasinya \textit{hosted} sehingga dapat diakses kapan pun.

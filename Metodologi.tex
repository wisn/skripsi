\tikzstyle{startstop} = [rectangle, rounded corners, minimum width=3cm, minimum height=1cm,text centered, draw=black, fill=red!30]
\tikzstyle{io} = [trapezium, trapezium left angle=70, trapezium right angle=110, minimum width=3cm, minimum height=1cm, text centered, draw=black, fill=blue!30]
\tikzstyle{process} = [rectangle, minimum width=3cm, minimum height=1cm, text centered, minimum width=3cm, draw=black, fill=orange!30]
\tikzstyle{decision} = [diamond, minimum width=3cm, minimum height=1cm, text centered, draw=black, fill=green!30]
\tikzstyle{arrow} = [thick,->,>=stealth]
\tikzstyle{cloud} = [draw, ellipse,fill=red!20, minimum height=2em]

\chapter{Metodologi dan Desain Sistem}

\section{Pembuatan \textit{Dataset}}

\textit{Dataset} untuk melakukan \textit{train} CCG sayangnya belum tersedia.
Karenanya, kita akan membuatkan \textit{dataset} secara semi otomatis dengan memanfaatkan POS
\textit{tagger} untuk bahasa Indonesia.
Secara formal, \textit{flowchart} untuk proses pembuatan \textit{dataset} dapat dilihat pada
Gambar \ref{flowchart:dataset:1}.

\begin{figure}[h!]\centering
	\begin{tikzpicture}[node distance=2cm]
		\node (start) [cloud] {Start};
		\node (in) [io, below of=start, yshift=0.5cm] {\textit{Input} teks bahasa Indonesia};
		\node (pro1) [process, below of=in, yshift=0.5cm] {POS \textit{tagging}};
		\node (pro2) [process, below of=pro1, yshift=0.5cm] {Jadikan \textit{queue}};
		\node (dec) [decision, below of=pro2, yshift=-1cm] {\textit{Queue} kosong?};
		\node (pro3) [process, below of=dec, yshift=-1cm] {Transformasi POS \textit{tag}};
		\node (out) [io, below of=pro3, yshift=0.5cm] {\textit{Output} CCG \textit{lexicon}};
		\node (stop) [cloud, below of=out, yshift=0.5cm] {Stop};
		\draw [arrow] (start) -- (in);
		\draw [arrow] (in) -- (pro1);
		\draw [arrow] (pro1) -- (pro2);
		\draw [arrow] (pro2) -- (dec);
		\draw [arrow] (dec) -- node[anchor=east]{Tidak} (pro3);
		\draw [arrow] (pro3) -- +(-3,0) |- (dec);
		\draw [arrow] (dec) -- node[anchor=south]{Ya} +(4,0) |- (out);
		\draw [arrow] (out) -- (stop);
	\end{tikzpicture}
	\caption{Alur kerja pembuatan \textit{dataset} untuk CCG \textit{lexicon}.}
	\label{flowchart:dataset:1}
\end{figure}

\subsection{\textit{Input} Teks Bahasa Indonesia}

Pada bagian \textit{input} dalam pembuatan \textit{dataset}, kita dapat mengambil teks bahasa Indonesia
dari Indonesian Treebank\footnotemark\ dan/atau dari beberapa contoh artikel yang terdapat di
\textit{website} Wikipedia Indonesia\footnotemark.
Indonesian Treebank terdapat setidaknya 1000 kalimat yang dirasa cukup untuk melakukan
\textit{training} menggunakan model MaxEnt.
Apa yang perlu dilakukan setelah mengambil \textit{input} adalah membersihkan bentuk \textit{tree}-nya
untuk kemudian diambilkan kalimat yang sebenarnya.
Caranya cukup sederhana yaitu dengan mengambil \textit{leaf} dari \textit{tree} tersebut kemudian
disatukan di suatu pengubah dengan tipe data \textit{string}.

Sebagai contoh, salah satu \textit{tree} yang terdapat di Indonesian Treebank dapat dilihat pada
Gambar \ref{treebank:tree:1}.
Apa yang dimaksud dengan \textit{leaf} pada \textit{tree} tersebut adalah $(Kera)$, $(untuk)$,
$(*)$, $(amankan)$, $(pesta\ olahraga)$.
Terkhusus untuk \textit{leaf} dengan bentuk spesial, seperti $(*)$, kita hilangkan sehingga teks
yang diperoleh dari \textit{tree} tersebut adalah \say{Kera untuk amankan pesta olahraga}.
Selanjutnya, contoh teks yang akan digunakan agar konsisten yaitu
\say{Pamungkas dan Setyo menyukai rendang}.


\begin{figure}\centering\small
	\begin{align*}
		&(NP\\
		&\ \ (NN\ (Kera))\\
		&\ \ (SBAR\\
		&\ \ \ \ (SC\ (untuk))\\
		&\ \ \ \ (S\ (NP-SBJ\ (*))\\
		&\ \ \ \ \ \ (VP\\
		&\ \ \ \ \ \ \ \ (VB\ (amankan))\\
		&\ \ \ \ \ \ \ \ (NP\ (NN\ (pesta\ olahraga)))))))
	\end{align*}
	\caption{Salah satu contoh \textit{tree} dalam Indonesian Treebank.}
	\label{treebank:tree:1}
\end{figure}

\footnotetext{github.com/famrashel/idn-treebank/blob/master/Indonesian\_Treebank.bracket}
\footnotetext{id.wikipedia.org}

\subsection{POS \textit{Tagging}}

Berdasarkan teks bahasa Indonesia yang telah diambil, kita manfaatkan \textit{tool} POS \textit{tagger}
bahasa Indonesia untuk mendapatkan \textit{lexical category} atomik untuk masing-masing token.
Dengan memanfaatkan POS \textit{tag} kita dapat membuat \textit{dataset} untuk CCG \textit{supertag}
lebih mudah dibandingkan dengan memberikan \textit{tag} CCG secara manual.
Ide dasarnya yaitu kita akan mentransformasikan POS \textit{tag} yang didapatkan menjadi CCG
\textit{supertag} berdasarkan aturan-aturan khusus yang telah ditentukan.
Sebagai contoh, kita dapat membuat aturan seperti mentransformasikan $\text{VB}$ menjadi
$\text{(S$\backslash$NP)/NP}$.
Dengan menggunakan contoh kalimat yang sama seperti di bagian sebelumnya,
yaitu \say{Pamungkas dan Setyo menyukai rendang}, setelah menggunakan POS \textit{tagger} bahasa
Indonesia kita dapatkan hasil sesuai dengan Gambar \ref{postag:1}.

\begin{figure}\centering
  \bgroup
  \catcode`!=\active \def!{\upshape}
  \catcode`?=\active \def?#1{\makebox[0pt]{#1}}
  \catcode`^=\active \def^#1{\footnotesize{#1}}
  \catcode`*=\active \def*#1{\scriptsize{#1}}
  \tabbedShortstack{
    !^Pamungkas & & !^dan & & !^Setyo & & !^menyukai & & !^rendang &\\
    \TABcline{1,3,5,7,9}
    !^{$\text{NNP}$} & &
      !^{$\text{CC}$} & &
      !^{$\text{NNP}$} & &
      !^{$\text{VB}$} & &
      !^{$\text{X}$} &
  }
	\egroup
	\caption{Kalimat contoh dengan POS \textit{tag}-nya.}
	\label{postag:1}
\end{figure}

\subsection{Transformasi POS \textit{Tag}}

Pada bagian proses transformasi POS \textit{tag} ke bentuk CCG \textit{supertag}-nya,
kita buatkan aturan-aturan transformasinya.
Aturan transformasi tersebut merupakan pemetaan berbasis aturan.
Sebagai contoh, kata \say{menyukai} memiliki POS \textit{tag} $\text{VB}$.
Anggap saja dalam aturan transformasi terdapat pemetaan $\text{VB} \vdash \text{(S$\backslash$NP)/NP}$.
Sehingga, kita dapatkan CCG \textit{supertag} untuk \say{menyukai} yaitu $\text{(S$\backslash$NP)/NP}$.
Kendati demikian, masih ada bagian yang belum kita dapatkan yaitu \textit{semantic representation}-nya.
Kita dapat menggunakan \textit{stemmer} bahasa Indonesia agar mendapatkan \textit{root words}
dari kata \say{menyukai} yaitu \say{suka}.
Langkah terakhirnya adalah membuatkan \textit{semantic representation}-nya berdasarkan
\textit{root words} yang telah diperoleh sehingga kita dapatkan fungsi lambdanya yaitu
$\lambda x.\lambda y.\ \text{suka}(y, x)$.

Selain menggunakan \textit{stemmer}, kita dapat menggunakan \textit{morphological analyzer}.
Untuk bahasa Indonesia, kita dapat menggunakan \textit{tool} bernama MorphInd\footnotemark.
\textit{Morphological analyzer} salah satu kegunaannya yaitu dapat menghasilkan \textit{root words}
sehingga dapat kita manfaatkan untuk membuat fungsi lambda.
Namun, kita dapat menggunakan MorphInd sebagai pelengkap POS \textit{tagger} untuk bahasa Indonesia.
Hal ini agar \textit{dataset} yang dibuatkan secara semi-otomatis ini dapat memiliki kualitas yang
baik.

\footnotetext{septinalarasati.com/morphind/}

% \section{Flowchart sistem}

% \begin{figure}[h!]
%     \centering
%     %Mulai menggambar Flowchart
% \begin{tikzpicture}[node distance=2cm]
% \node (start) [cloud] {Start};
% \node (in1) [io, below of=start] {Input};
% \node (pro1) [process, below of=in1] {Process 1};
% \node (dec1) [decision, below of=pro1] {Decision 1};
% \node (pro2a) [process, below of=dec1, yshift=-0.5cm] {Process 2a};
% \node (pro2b) [process, right of=dec1, xshift=2cm] {Process 2b};
% \node (out1) [io, below of=pro2a] {Output};
% \node (stop) [cloud, below of=out1] {Stop};
% \draw [arrow] (start) -- (in1);
% \draw [arrow] (in1) -- (pro1);
% \draw [arrow] (pro1) -- (dec1);
% \draw [arrow] (dec1) -- (pro2a);
% \draw [arrow] (dec1) -- (pro2b);
% \draw [arrow] (pro2b) |- (pro1);
% \draw [arrow] (pro2a) -- (out1);
% \draw [arrow] (out1) -- (stop);
% \end{tikzpicture}
%     \caption{Caption flowchart}
%     \label{figflow}
% \end{figure}

% \section{Algoritma}
%  Atau dalam bentuk algoritma seperti contoh pada Algoritma \ref{Algo:FVDM} berikut ini:
 

% \begin{algorithm}
%  \begin{algorithmic}[1]
%     \Procedure{FVDM}{$Tfinal, \Delta t$}
%     \State \text{Start}
% 	\State \textbf{For }$n=1:N$\textbf{ do} \Comment{Pemberian nilai awal}
% 	\State \hspace{0.5cm} Input nilai $x[n]$
% 	\State \hspace{0.5cm} Input nilai $v[n]$
% 	\State \textbf{EndFor}
% 	\State \text{time=0}
% 	\While{$time < Tfinal$}
% 	\State \hspace{0.5cm} $time=time +\Delta t$
% 	\State \hspace{0.5cm} Hitung jarak bamper menggunakan rumus  untuk $n=2,\cdots,N$ 
% 	\State \hspace{0.5cm} \textbf{If}( $S(n) \leq 0 m)$ \textbf{then return End If}.
% 	\State \hspace{0.5cm} Tentukan $\lambda$ menggunakan.
% 	\State \hspace{0.5cm} Hitung kecepatan optimal $v_o(t)$ menggunakan.
% 	\State \hspace{0.5cm} Hitung percepatan $a_n(time)$ menggunakan .
% 	\State \hspace{0.5cm} Hitung kecepatan baru dengan $v_n(time)=v_n(time-\Delta t) + a_n(time) \Delta t$.
% 	\State \hspace{0.5cm} Hitung posisi baru dengan $x_n(time)=x_n(time-\Delta t) + v_n(time) \Delta t$.
% 	\State \hspace{0.5cm} \textbf{If}( $\Delta v \leq 10^{-5} \&\& a_n(time)\leq 10^{-5})$ \textbf{then}  
%     \State \hspace{0.5cm} \hspace{0.5cm} \text{OUTPUT }Cetak hasil data $a_n, v_n, x_n$.
% 	\State \hspace{0.5cm} \hspace{0.5cm} \textbf{return}.
% 	\State \hspace{0.5cm} \textbf{End If}.
% 	\EndWhile
% 	\State \text{End}
%  \EndProcedure
%  \end{algorithmic}
%  \caption{Prosedur simulasi dinamika lalu lintas menggunakan FVDM.}\label{Algo:FVDM}
% \end{algorithm}

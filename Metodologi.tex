\tikzstyle{startstop} = [rectangle, rounded corners, minimum width=3cm, minimum height=1cm,text centered, draw=black, fill=red!30]
\tikzstyle{io} = [trapezium, trapezium left angle=70, trapezium right angle=110, minimum width=3cm, minimum height=1cm, text centered, draw=black, fill=blue!30]
\tikzstyle{process} = [rectangle, minimum width=3cm, minimum height=1cm, text centered, minimum width=3cm, draw=black, fill=orange!30]
\tikzstyle{decision} = [diamond, minimum width=3cm, minimum height=1cm, text centered, draw=black, fill=green!30]
\tikzstyle{arrow} = [thick,->,>=stealth]
\tikzstyle{cloud} = [draw, ellipse,fill=red!20, minimum height=2em]

\chapter{Metodologi dan Desain Sistem}

\section{Pembuatan Dataset}

\textit{Dataset} untuk melakukan \textit{train} CCG sayangnya belum tersedia.
Karenanya, kita akan membuatkan \textit{dataset} secara semi otomatis dengan memanfaatkan POS
\textit{tagger} untuk bahasa Indonesia.
Secara formal, \textit{flowchart} untuk proses pembuatan \textit{dataset} dapat dilihat pada
Gambar \ref{flowchart:dataset:1}.

\begin{figure}[h!]\centering
	\begin{tikzpicture}[node distance=2cm]
		\node (start) [cloud] {Start};
		\node (in) [io, below of=start, yshift=0.5cm] {\textit{Input} teks bahasa Indonesia};
		\node (pro1) [process, below of=in, yshift=0.5cm] {POS \textit{tagging}};
		\node (pro2) [process, below of=pro1, yshift=0.5cm] {Jadikan \textit{queue}};
		\node (dec) [decision, below of=pro2, yshift=-1cm] {\textit{Queue} kosong?};
		\node (pro3) [process, below of=dec, yshift=-1cm] {Transformasi POS \textit{tag}};
		\node (out) [io, below of=pro3, yshift=0.5cm] {\textit{Output} CCG \textit{supertag}};
		\node (stop) [cloud, below of=out, yshift=0.5cm] {Stop};
		\draw [arrow] (start) -- (in);
		\draw [arrow] (in) -- (pro1);
		\draw [arrow] (pro1) -- (pro2);
		\draw [arrow] (pro2) -- (dec);
		\draw [arrow] (dec) -- node[anchor=east]{Tidak} (pro3);
		\draw [arrow] (pro3) -- +(-3,0) |- (dec);
		\draw [arrow] (dec) -- node[anchor=south]{Ya} +(4,0) |- (out);
		\draw [arrow] (out) -- (stop);
	\end{tikzpicture}
	\caption{Alur kerja pembuatan \textit{dataset} untuk CCG \textit{supertag}.}
	\label{flowchart:dataset:1}
\end{figure}

\subsection{Input Teks Bahasa Indonesia}

Pada bagian \textit{input} dalam pembuatan \textit{dataset}, kita dapat mengambil teks bahasa Indonesia
dari Indonesian Treebank\footnotemark[1]\ dan/atau dari beberapa contoh artikel yang terdapat di
\textit{website} Wikipedia Indonesia\footnotemark[2].
Indonesian Treebank terdapat setidaknya 1000 kalimat yang dirasa cukup untuk melakukan
\textit{training} menggunakan model MaxEnt.
Apa yang perlu dilakukan setelah mengambil \textit{input} adalah membersihkan bentuk \textit{tree}-nya
untuk kemudian diambilkan kalimat yang sebenarnya.
Caranya cukup sederhana yaitu dengan mengambil \textit{leaf} dari \textit{tree} tersebut kemudian
disatukan di suatu pengubah dengan tipe data \textit{string}.

Sebagai contoh, salah satu \textit{tree} yang terdapat di Indonesian Treebank dapat dilihat pada
Gambar \ref{treebank:tree:1}.
Apa yang dimaksud dengan \textit{leaf} pada \textit{tree} tersebut adalah $(Kera)$, $(untuk)$,
$(*)$, $(amankan)$, $(pesta\ olahraga)$.
Terkhusus untuk \textit{leaf} dengan bentuk spesial, seperti $(*)$, kita hilangkan sehingga teks
yang diperoleh dari \textit{tree} tersebut adalah \say{Kera untuk amankan pesta olahraga}.
Selanjutnya, contoh teks yang akan digunakan agar konsisten yaitu
\say{Pamungkas dan Setyo menyukai rendang}.


\begin{figure}\centering\small
	\begin{align*}
		&(NP\\
		&\ \ (NN\ (Kera))\\
		&\ \ (SBAR\\
		&\ \ \ \ (SC\ (untuk))\\
		&\ \ \ \ (S\ (NP-SBJ\ (*))\\
		&\ \ \ \ \ \ (VP\\
		&\ \ \ \ \ \ \ \ (VB\ (amankan))\\
		&\ \ \ \ \ \ \ \ (NP\ (NN\ (pesta\ olahraga)))))))
	\end{align*}
	\caption{Salah satu contoh \textit{tree} dalam Indonesian Treebank.}
	\label{treebank:tree:1}
\end{figure}

\footnotetext[1]{github.com/famrashel/idn-treebank/blob/master/Indonesian\_Treebank.bracket}
\footnotetext[2]{id.wikipedia.org}

\subsection{POS Tagging}

Berdasarkan teks bahasa Indonesia yang telah diambil, kita manfaatkan \textit{tool} POS \textit{tagger}
bahasa Indonesia untuk mendapatkan \textit{lexical category} atomik untuk masing-masing token.
Dengan memanfaatkan POS \textit{tag} kita dapat membuat \textit{dataset} untuk CCG \textit{supertag}
lebih mudah dibandingkan dengan memberikan \textit{tag} CCG secara manual.
Ide dasarnya yaitu kita akan mentransformasikan POS \textit{tag} yang didapatkan menjadi CCG
\textit{supertag} berdasarkan aturan-aturan khusus yang telah ditentukan.
Sebagai contoh, kita dapat membuat aturan seperti mentransformasikan $\text{VB}$ menjadi
$\text{(S$\backslash$NP)/NP}$.
Dengan menggunakan contoh kalimat yang sama seperti di bagian sebelumnya,
yaitu \say{Pamungkas dan Setyo menyukai rendang}, setelah menggunakan POS \textit{tagger} bahasa
Indonesia kita dapatkan hasil sesuai dengan Gambar \ref{postag:1}.

\begin{figure}\centering
  \bgroup
  \catcode`!=\active \def!{\upshape}
  \catcode`?=\active \def?#1{\makebox[0pt]{#1}}
  \catcode`^=\active \def^#1{\footnotesize{#1}}
  \catcode`*=\active \def*#1{\scriptsize{#1}}
  \tabbedShortstack{
    !^Pamungkas & & !^dan & & !^Setyo & & !^menyukai & & !^rendang &\\
    \TABcline{1,3,5,7,9}
    !^{$\text{NNP}$} & &
      !^{$\text{CC}$} & &
      !^{$\text{NNP}$} & &
      !^{$\text{VB}$} & &
      !^{$\text{X}$} &
  }
	\egroup
	\caption{Kalimat contoh dengan POS \textit{tag}-nya.}
	\label{postag:1}
\end{figure}

\subsection{Transformasi POS Tag}

Pada bagian proses transformasi POS \textit{tag} ke bentuk CCG \textit{supertag}-nya,
kita buatkan aturan-aturan transformasinya.
Aturan transformasi tersebut merupakan pemetaan berbasis aturan.
Sebagai contoh, kata \say{menyukai} memiliki POS \textit{tag} $\text{VB}$.
Anggap saja dalam aturan transformasi terdapat pemetaan $\text{VB} \vdash \text{(S$\backslash$NP)/NP}$.
Sehingga, kita dapatkan CCG \textit{supertag} untuk \say{menyukai} yaitu $\text{(S$\backslash$NP)/NP}$.
Kendati demikian, masih ada bagian yang belum kita dapatkan yaitu \textit{semantic representation}-nya.
Kita dapat menggunakan \textit{stemmer} bahasa Indonesia agar mendapatkan \textit{root words}
dari kata \say{menyukai} yaitu \say{suka}.
Langkah terakhirnya adalah membuatkan \textit{semantic representation}-nya berdasarkan
\textit{root words} yang telah diperoleh sehingga kita dapatkan fungsi lambdanya yaitu
$\lambda x.\lambda y.\ \text{suka}(y, x)$.

Selain menggunakan \textit{stemmer}, kita dapat menggunakan \textit{morphological analyzer}.
Untuk bahasa Indonesia, kita dapat menggunakan \textit{tool} bernama MorphInd\footnotemark[3].
\textit{Morphological analyzer} salah satu kegunaannya yaitu dapat menghasilkan \textit{root words}
sehingga dapat kita manfaatkan untuk membuat fungsi lambda.
Namun, kita dapat menggunakan MorphInd sebagai pelengkap POS \textit{tagger} untuk bahasa Indonesia.
Hal ini agar \textit{dataset} yang dibuatkan secara semi-otomatis ini dapat memiliki kualitas yang
baik.

\footnotetext[3]{septinalarasati.com/morphind}

\subsection{Output CCG Supertag}

Keluaran dari bagian pembuatan \textit{dataset} ini adalah sebuah berkas JSON
(JavaScript Object Notation) berisi CCG \textit{supertag} lengkap dan beberapa berkas JSON dari CCG
\textit{supertag} yang lemanya dikelompokkan berasarkan alfabet.
Adapun format JSON dari \textit{dataset} yang disimpan dalam berkas tersebut dapat dilihat pada Gambar
\ref{dataset:format}.
Untuk setiap kalimat dalam berkas tersebut direpresentasikan oleh dua objek yaitu
(1) \say{tokens} berupa daftar token, dan
(2) \say{supertags} berupa daftar \textit{supertag} untuk token ke-$(i, j)$ dimana $1 \leq i \leq n$
dan $1 \leq j \leq m$ dalam sistem \textit{$1$-indexed array}.
Sebagai contoh, dengan kalimat yang sama dengan sebelum-sebelumnya, dapat dilihat di Gambar
\ref{dataset:example}.
Demikian itu, kita dapat mengambil \textit{dataset} secara lengkap dengan CCG \textit{supertag}-nya.

\begin{figure}\centering
	\bgroup
	\catcode`!=\active \def!{\upshape}
	\tabbedShortstack[l]{
		!{[} & & &\\
			& !{\{} & &\\
			& & !{\say{tokens}:} &
				!{[\say{$token_{1,1}$}, \say{$token_{1,2}$}, $\dots$, \say{$token_{1,m}$}],}\\
			& & !{\say{supertags}:} &
				!{[\say{$supertag_{1,1}$}, \say{$supertag_{1,2}$}, $\dots$, \say{$supertag_{1,m}$}]}\\
			& !{\}}, & &\\
			& !{\{} & &\\
			& & !{\say{tokens}:} &
				!{[\say{$token_{2,1}$}, \say{$token_{2,2}$}, $\dots$, \say{$token_{2,m}$}],}\\
			& & !{\say{supertags}:} &
				!{[\say{$supertag_{2,1}$}, \say{$supertag_{2,2}$}, $\dots$, \say{$supertag_{2,m}$}]}\\
			& !{\}}, & &\\
			& & &\\
			& & &\\
			& $\vdots$ & &\\
			& & &\\
			& & &\\
			& !{\{} & &\\
			& & !{\say{tokens}:} &
				!{[\say{$token_{n,1}$}, \say{$token_{n,2}$}, $\dots$, \say{$token_{n,m}$}],}\\
			& & !{\say{supertags}:} &
				!{[\say{$supertag_{n,1}$}, \say{$supertag_{n,2}$}, $\dots$, \say{$supertag_{n,m}$}]}\\
			& !{\}} & &\\
		!{]} & & &
	}
	\egroup
	\caption{Format JSON \textit{dataset} yang disimpan di dalam berkas.}
	\label{dataset:format}
\end{figure}

\begin{figure}\centering
	\bgroup
	\catcode`!=\active \def!{\upshape}
	\catcode`?=\active \def?#1{\makebox[0pt]{#1}}
	\tabbedShortstack[l]{
		!{[} & & &\\
			& !{\{} & &\\
			& & !{\say{tokens}: [} &\\
			& & !{\ \ \ \ \say{Pamungkas},} &\\
			& & !{\ \ \ \ \say{dan},} &\\
			& & !{\ \ \ \ \say{Setyo},} &\\
			& & !{\ \ \ \ \say{menyukai},} &\\
			& & !{\ \ \ \ \say{rendang},} &\\
			& & !{\ \ \ \ \say{.}} &\\
			& & !{],} &\\
			& & !{\say{supertags}: [} &\\
			& & !{\ \ \ \ \say{$\text{NP: \so{pamungkas}}$},} &\\
			& & !{\ \ \ \ \say{$\text{CONJ: $\lambda x.\lambda y.\lambda f.\ (f\ x) \land (f\ y)$}$},} &\\
			& & !{\ \ \ \ \say{$\text{NP: \so{setyo}}$},} &\\
			& & !{\ \ \ \ \say{$\text{(S$\backslash$NP)/NP: $\lambda x.\lambda y.\ suka(y, x)$}$},} &\\
			& & !{\ \ \ \ \say{$\text{NP: \so{rendang}}$},} &\\
			& & !{\ \ \ \ \say{$\text{Z}$}} &\\
			& & !{]} &\\
			& !{\}} & &\\
		!{]} & & &
	}
	\egroup
	\caption{Contoh isi dari berkas \textit{dataset} dalam format JSON.}
	\label{dataset:example}
\end{figure}

\section{Melatih Model Supertagger}

Model yang akan digunakan oleh \textit{supertagger} ini adalah Maximum Entropy (MaxEnt).
Model ini dipilih karena ketersediaan \textit{dataset} bahasa Indonesia yang masih sangat kurang.
Selain itu, model MaxEnt digunakan di sebuah riset yang dilakukan oleh Stephen Clark dalam
pembuatan \textit{supertagger}-nya.
Bahkan, performansi \textit{supertagger} yang dikembangkan sangat baik.
Riset tersebut pada intinya membuktikan bahwasannya MaxEnt dapat digunakan di \textit{supertagger}
juga meskipun pada peruntukannya MaxEnt dibuat untuk POS \textit{tagger}.
Tentunya terdapat beberapa penyesuaian yang harus dilakukan.
Salah satunya adalah formula probabiltas yang digunakan.
Formula yang telah disesuaikan tersebut dapat dilihat di persamaan \ref{maxent:equation:2}.

\chapter{Pendahuluan}
\section{Latar Belakang}

Riset pemrosesan bahasa natural untuk bahasa Indonesia saat ini terbilang sedikit.
Bahkan, masih banyak area riset yang belum tersentuh seperti contohnya
\textit{combinatory categorial grammar} (CCG).
CCG merupakan sebuah formalisme tatabahasa yang dapat dimanfaatkan untuk membangun CCG \textit{parser}.
CCG \textit{parser} dapat digunakan untuk mendapatkan berbagai macam informasi dari suatu kalimat.
Sebagai contoh, CCG \textit{parser} dapat mem-\textit{parse} kalimat
"sebutkan negara-negara yang bertetangga dengan Indonesia" ke dalam bentuk formal yang dapat dipahami
oleh komputer yaitu $\lambda{(x). negara(x) \land bertetangga(x, Indonesia)}$.
Bentuk formal tersebut merupakan \textit{lambda calculus}
(umumnya bentuk formal yang digunakan adalah \textit{combinatory logic}) yang kemudian dapat diproses
oleh komputer strukturnya dan akan mendapatkan \textit{query} untuk mencari $x$ yang berupa
suatu negara dan memiliki ketetanggaan dengan Indonesia.

CCG \textit{parser} membutuhkan CCG \textit{lexicon} (atau dikenal juga sebagai CCG \textit{supertag})
untuk dapat melakukan tugasnya. Sejauh ini belum ditemukan adanya riset mengenai CCG untuk
bahasa Indonesia termasuk pada tahap awalnya yaitu pembentukan CCG \textit{supertag}.
Demikian itu, tugas akhir ini diharapkan dapat menjadi inisiator riset untuk area CCG dalam
pemrosesan bahasa alami yaitu dengan membangun perangkat lunak untuk menghasilkan CCG \textit{supertag}
bahasa Indonesia. Proses yang menghasilkan \textit{supertag} tersebut bernama \textit{supertagging}
adapun perangkat lunaknya bernama \textit{supertagger}.

\section{Perumusan Masalah}
Berikut adalah rumusan masalah yang akan diangkat:
\begin{enumerate}
    \item Mengapa CCG \textit{supertagger} diperlukan?
    \item Apa saja yang harus dipersiapkan untuk membangun CCG \textit{supertagger}?
    \item Bagaimana proses membangun CCG \textit{supertagger}?
\end{enumerate}
\section{Tujuan}
Berikut adalah tujuan yang diharapkan dapat dicapai oleh tugas akhir ini:
\begin{enumerate}
    \item Mengenalkan alternatif metode yang dapat digunakan dalam pemrosesan bahasa alami untuk
      bahasa Indonesia.
    \item Merilis CCG \textit{supertagger} pertama untuk bahasa Indonesia.
    \item Membuka peluang riset untuk CCG \textit{parser} bahasa Indonesia.
\end{enumerate}
% \section{Batassan Masalah}
% Hipotesis dari tulisan ini adalah
% \begin{enumerate}
%     \item Masalah timbul karena A;
%     \item Hasil numeriknya menuju $x \rightarrow \infty$
% \end{enumerate}

% \section{Rencana Kegiatan}
% Rencana kegitana yang akan saya lakukan adalah sebagia berikut:
% \begin{itemize}
%     \item Studi literatur
%     \item Memeriksa hasil
% \end{itemize}
% \section{Jadwal Kegiatan}
% The table \ref{Novella} is an example of referenced \LaTeX elements. Laporan proposal ini akan dijadwalkan sesuai dengan tabel yang diberikna berikutnya. 

 
% \begin{table}[h!]
%   \centering
%     \caption{Jadwal kegiatan proposal tugas akhir}
%   \label{Novella}
%   \begin{tabular}{|c|m{2.5cm}|m{0.01cm}|m{0.01cm}|m{0.01cm}|m{0.01cm}|m{0.01cm}|m{0.01cm}|m{0.01cm}|m{0.01cm}|m{0.01cm}|m{0.01cm}|m{0.01cm}|m{0.01cm}|m{0.01cm}|m{0.01cm}|m{0.01cm}|m{0.01cm}|m{0.01cm}|m{0.01cm}|m{0.01cm}|m{0.01cm}|m{0.01cm}|m{0.01cm}|m{0.01cm}|m{0.01cm}|}
%     \hline
%     \multirow{2}{*}{\textbf{No}} & \multirow{2}{*}{\textbf{Kegiatan}} & \multicolumn{24}{|c|}{\textbf{Bulan ke-}} \\
%     \hhline{~~------------------------}
%     {} & {} & \multicolumn{4}{|c|}{\textbf{1}} & \multicolumn{4}{|c|}{\textbf{2}} & \multicolumn{4}{|c|}{\textbf{3}} & \multicolumn{4}{|c|}{\textbf{4}} & \multicolumn{4}{|c|}{\textbf{5}} & \multicolumn{4}{|c|}{\textbf{6}}\\
%     \hline
%     1 & Studi Literatur & \cellcolor{blue!25} & \cellcolor{blue!25} & \cellcolor{blue!25} & \cellcolor{blue!25}& \cellcolor{blue!25} & \cellcolor{blue!25} & \cellcolor{blue!25} & \cellcolor{blue!25}& \cellcolor{blue!25} & \cellcolor{blue!25} & \cellcolor{blue!25} & \cellcolor{blue!25}& \cellcolor{blue!25} & \cellcolor{blue!25} & \cellcolor{blue!25} & \cellcolor{blue!25}& \cellcolor{blue!25} & \cellcolor{blue!25} & \cellcolor{blue!25} & \cellcolor{blue!25}& \cellcolor{blue!25} & \cellcolor{blue!25} & \cellcolor{blue!25} & \cellcolor{blue!25}\\
%     \hline
%     2 & Pengumpulan Data & \cellcolor{blue!25} & \cellcolor{blue!25} & \cellcolor{blue!25} & \cellcolor{blue!25} & {} & {} & {} & {} & {} & {} & {} & {}& {} & {} & {} & {}& {} & {} & {} & {}& {} & {} & {} & {}\\
%     \hline
%     3 & Analisis dan Perancangan Sistem &  {} & {} & {} & {}  & \cellcolor{blue!25} & \cellcolor{blue!25} & \cellcolor{blue!25} & \cellcolor{blue!25} & \cellcolor{blue!25} & \cellcolor{blue!25} & \cellcolor{blue!25} & \cellcolor{blue!25} & {} & {} & {} & {}& {} & {} & {} & {}& {} & {} & {} & {}\\
%     \hline
%     4 & Implementasi Sistem &  {} & {} & {} & {} & {} & {} & {} & {}& \cellcolor{blue!25} & \cellcolor{blue!25} & \cellcolor{blue!25} & \cellcolor{blue!25} & \cellcolor{blue!25} & \cellcolor{blue!25} & \cellcolor{blue!25} & \cellcolor{blue!25} & {} & {} & {} & {}& {} & {} & {} & {}\\
%     \hline
%     5 & Analisa Hasil Implementasi &  {} & {} & {} & {} & {} & {} & {} & {}& {} & {} & {} & {} & \cellcolor{blue!25} & \cellcolor{blue!25} & \cellcolor{blue!25} & \cellcolor{blue!25} & \cellcolor{blue!25} & \cellcolor{blue!25} & \cellcolor{blue!25} & \cellcolor{blue!25} & {} & {} & {} & {}\\
%     \hline
%     6 & Penulisan Laporan & {} & {} & {} & {} & \cellcolor{blue!25} & \cellcolor{blue!25} & \cellcolor{blue!25} & \cellcolor{blue!25}& \cellcolor{blue!25} & \cellcolor{blue!25} & \cellcolor{blue!25} & \cellcolor{blue!25}& \cellcolor{blue!25} & \cellcolor{blue!25} & \cellcolor{blue!25} & \cellcolor{blue!25}& \cellcolor{blue!25} & \cellcolor{blue!25} & \cellcolor{blue!25} & \cellcolor{blue!25}& \cellcolor{blue!25} & \cellcolor{blue!25} & \cellcolor{blue!25} & \cellcolor{blue!25}\\
%     \hline
%   \end{tabular}

% \end{table}

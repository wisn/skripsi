\chapter*{Abstrak}

Dalam pemrosesan bahasa alami, combinatory categorial grammar (CCG) merupakan salah satu
formalisme tata bahasa yang dapat digunakan untuk membangun sebuah \textit{parser} yang umumnya
dikenal sebagai CCG \textit{parser}.
CCG \textit{parser} dapat digunakan untuk berbagai macam keperluan dalam pemrosesan bahasa alami.
Sebagai contoh, CCG \textit{parser} dapat digunakan untuk memperoleh informasi
(\textit{information extraction}) dari suatu kalimat yang kemudian membentuk sebuah \textit{query}.
Agar dapat membangun CCG \textit{parser} ataupun \textit{tools} lainnya, \textit{dataset} CCG
yang memuat \textit{lexicon} dibutuhkan. Saat Tugas Akhir ini ditulis \textit{dataset} CCG untuk
bahasa Indonesia belum tersedia. Demikian itu, CCGtown dibangun sebagai \textit{annotation tool}
yang dapat digunakan untuk membangun \textit{dataset} CCG.
CCGtown merupakan \textit{open source graphical annotation tool} berbasis web yang dirancang
dengan fokus mempercepat serta mempermudah proses penganotasian.
Dengan fitur \textit{generate} CCG \textit{derivation} serta \textit{auto-assign} CCG
\textit{category} kegiatan repetitif dalam melakukan anotasi akhirnya berkurang.
Dengan antarmuka (UI/UX) yang cukup baik menjadikan CCGtown alat anotasi yang mudah digunakan
(\textit{user friendly}) oleh berbagai kalangan pengguna.

\vspace{0.5 cm}
\begin{flushleft}
{\textbf{Kata Kunci:}
  natural language processing, combinatory categorial grammar, annotation tool
}
\end{flushleft}

\chapter*{Abstrak}

% Dalam pemrosesan bahasa alami, combinatory categorial grammar (CCG) merupakan salah satu
% formalisme tata bahasa yang dapat digunakan untuk membangun sebuah \textit{parser} yang umumnya
% dikenal sebagai CCG \textit{parser}.
% CCG \textit{parser} dapat digunakan untuk berbagai macam keperluan dalam pemrosesan bahasa alami.
% Sebagai contoh, CCG \textit{parser} dapat digunakan untuk memperoleh informasi
% (\textit{information extraction}) dari suatu kalimat yang kemudian membentuk sebuah \textit{query}.
% Agar dapat bekerja, CCG \textit{parser} membutuhkan CCG \textit{lexicon}.
% CCG \textit{lexicon} diperoleh dari proses yang bernama \textit{supertagging}.
% \textit{Supertagging} adalah proses pelabelan suatu token kata terhadap \textit{supertag}-nya.
% Perangkat lunak yang melakukan \textit{supertagging} disebut sebagai \textit{supertagger}.
% Demikian itu, \textit{supertagging} merupakan langkah pertama yang perlu dilakukan sebelum
% membangun sebuah CCG \textit{parser}.
% \textit{Supertagger} yang dibangun dalam tugas akhir ini dimaksudkan sebagai produsen CCG
% \textit{lexicon} bahasa Indonesia untuk riset-riset yang berkenaan dengan CCG di masa yang akan datang.

TBA.

\vspace{0.5 cm}
\begin{flushleft}
{\textbf{Kata Kunci:}
  natural language processing, combinatory categorial grammar, annotation tool
}
\end{flushleft}

\chapter{Pendahuluan}
\section{Latar Belakang}

Riset pemrosesan bahasa natural untuk bahasa Indonesia saat ini terbilang sedikit.
Bahkan, masih banyak area riset yang belum tersentuh seperti contohnya
\textit{combinatory categorial grammar} (CCG).
Riset mengenai CCG untuk bahasa Inggris sudah cukup matang.
Adapun untuk bahasa lainnya, seperti bahasa Vietnam, sudah memulai untuk menggunakan CCG di dalam
penelitiannya \cite{nguyen2019vietnamese}.
Umumnya terdapat dua cara yang paling sering digunakan untuk mengembangkan CCG \textit{supertagger}
maupun CCG \textit{parser} bahasa lokal yaitu (1) membangun \textit{dataset} CCG \textit{supertag}
secara manual maupun semi-otomatis atau (2) melakukan transfer \textit{dataset} dari CCGBank
ke dalam bahasa lokal dengan cara melakukan alih bahasa dan bila perlu melakukan penyesuaian untuk
\textit{supertag}-nya \cite{hockenmaier-steedman-2007-ccgbank}.
Berhubung riset CCG di Indonesia masih kurang peminatnya, tugas akhir ini diangkat untuk menimbulkan
minat dengan cara menyediakan \textit{tool} esensial yaitu CCG \textit{supertagger}.

% CCG merupakan formalisme tata bahasa yang salah satu manfaatnya adalah untuk memperoleh informasi
% (\textit{information extraction}) dari suatu kalimat.
% Informasi tersebut diperoleh setelah melakukan \textit{parsing} berdasarkan formalisme CCG dengan
% menggunakan perangkat lunak bernama CCG \textit{parser}.
% Untuk dapat melakukan \textit{parsing}, CCG \textit{parser} membutuhkan CCG \textit{lexicon}
% yang mengandung bentuk formal dari suatu token kata.
% Bentuk formal yang dimaksud adalah \textit{category} dalam \textit{category theory}.
% CCG \textit{lexicon} diperoleh dari proses pelabelan suatu token kata terhadap bentuk formalnya yang
% mana dikenal sebagai \textit{supertagging}.
% Proses \textit{supertagging} akan menghasilkan \textit{supertag} yang kemudian disebut sebagai CCG
% \textit{supertag} karena formalisme yang digunakan adalah formalisme CCG.
% Dalam hal ini, CCG \textit{supertag} adalah CCG \textit{lexicon} itu sendiri.

Tugas akhir dengan judul \say{\Title} berusaha untuk membangun versi awal dari CCG
\textit{supertagger} untuk bahasa Indonesia yang mana harapannya dapat menjadi inisiator riset
pemrosesan bahasa natural dengan tema CCG sehingga ke depannya akan ada lebih banyak riset
mengenai CCG yang tersedia.
Acuan utama \textit{paper} yang digunakan dalam membangun \textit{supertagger} ini adalah
\textit{paper} yang dipublikasikan oleh Stephen Clark pada tahun 2002 \cite{clark-2002-supertagging}.
\textit{Supertagger} yang dimaksud dalam tugas akhir ini akan dibangun dengan menggunakan model
Maximum Entropy (MaxEnt) dan implementasinya akan ditulis dalam bahasa pemrograman Haskell.
Model MaxEnt digunakan karena keterbatasan \textit{dataset} untuk melakukan \textit{learning}.
Selain itu, model MaxEnt sudah terbukti dapat digunakan untuk membangun CCG \textit{supertagger}
\cite{clark-2002-supertagging} berdasarkan \textit{paper} yang dipublikasikan oleh Stephen Clark.
Adapun bahasa pemrograman Haskell digunakan karena abstraksi bahasanya yang sangat mendekati
\textit{category theory} serta kemampuannya yang sangat baik dalam pemrosesan data.

\section{Perumusan Masalah}
Rumusan masalah yang akan diangkat yaitu:
\begin{enumerate}
    \item Bagaimana proses pembangunan CCG \textit{supertagger} untuk bahasa Indonesia?
    \item Bagaimana dengan akurasi CCG \textit{supertagger} apabila dibandingkan dengan
      POS \textit{tagger} untuk bahasa Indonesia?
\end{enumerate}
\section{Tujuan}
Tujuan yang diharapkan dapat tercapai oleh tugas akhir ini yaitu:
\begin{enumerate}
    \item Membangun dan merilis CCG \textit{supertagger} pertama untuk bahasa Indonesia.
    \item Membuka peluang riset untuk CCG \textit{parser} bahasa Indonesia.
\end{enumerate}
\section{Batasan Masalah}
Hipotesis dari tugas akhir ini yaitu:
\begin{enumerate}
    \item Memberikan label CCG untuk proses \textit{training} merupakan permasalahan utama dari tugas
      akhir ini.
    \item \textit{Supertagger} yang akan dibangun kemungkinan besar memiliki akurasi yang cenderung
      rendah.
    \item CCG \textit{supertag} sudah dapat digunakan oleh CCG \textit{parser} bahasa Indonesia
      (apabila ada).
\end{enumerate}

\section{Rencana Kegiatan}
Rencana kegiatan yang akan dilakukan adalah sebagai berikut:
\begin{itemize}
    \item Studi literatur
    \item Studi \textit{tools} yang tersedia
    \item Studi bahasa pemrograman yang akan digunakan
    \item Perancangan sistem \textit{supertagger}
    \item Membangun \textit{supertagger}
    \item Memeriksa hasil
\end{itemize}

\section{Jadwal Kegiatan}
Laporan proposal ini akan dijadwalkan sesuai dengan tabel \ref{ScheduleTable}. 
\begin{table}[h!]
  \centering
    \caption{Jadwal kegiatan proposal tugas akhir.}
  \label{ScheduleTable}
  \begin{tabular}{|c|m{2.5cm}|m{0.01cm}|m{0.01cm}|m{0.01cm}|m{0.01cm}|m{0.01cm}|m{0.01cm}|m{0.01cm}|m{0.01cm}|m{0.01cm}|m{0.01cm}|m{0.01cm}|m{0.01cm}|m{0.01cm}|m{0.01cm}|m{0.01cm}|m{0.01cm}|m{0.01cm}|m{0.01cm}|m{0.01cm}|m{0.01cm}|m{0.01cm}|m{0.01cm}|m{0.01cm}|m{0.01cm}|}
    \hline
    \multirow{2}{*}{\textbf{No}} & \multirow{2}{*}{\textbf{Kegiatan}} & \multicolumn{24}{|c|}{\textbf{Bulan ke-}} \\
    \hhline{~~------------------------}
    {} & {} & \multicolumn{4}{|c|}{\textbf{1}} & \multicolumn{4}{|c|}{\textbf{2}} & \multicolumn{4}{|c|}{\textbf{3}} & \multicolumn{4}{|c|}{\textbf{4}} & \multicolumn{4}{|c|}{\textbf{5}} & \multicolumn{4}{|c|}{\textbf{6}}\\
    \hline
    1 & Studi Literatur & \cellcolor{blue!25} & \cellcolor{blue!25} & \cellcolor{blue!25} & \cellcolor{blue!25}& \cellcolor{blue!25} & \cellcolor{blue!25} & \cellcolor{blue!25} & \cellcolor{blue!25}& \cellcolor{blue!25} & \cellcolor{blue!25} & \cellcolor{blue!25} & \cellcolor{blue!25}& \cellcolor{blue!25} & \cellcolor{blue!25} & \cellcolor{blue!25} & \cellcolor{blue!25}& \cellcolor{blue!25} & \cellcolor{blue!25} & \cellcolor{blue!25} & \cellcolor{blue!25}& \cellcolor{blue!25} & \cellcolor{blue!25} & \cellcolor{blue!25} & \cellcolor{blue!25}\\
    \hline
    2 & Studi \textit{Tools} yang Tersedia & \cellcolor{blue!25} & \cellcolor{blue!25} & \cellcolor{blue!25} & \cellcolor{blue!25} &  \cellcolor{blue!25} &  \cellcolor{blue!25} &  \cellcolor{blue!25} &  \cellcolor{blue!25} & {} & {} & {} & {}& {} & {} & {} & {}& {} & {} & {} & {}& {} & {} & {} & {}\\
    \hline
    3 & Studi Bahasa Pemrograman & {} & {} & {} & {} & \cellcolor{blue!25} & \cellcolor{blue!25} & \cellcolor{blue!25} & \cellcolor{blue!25} & {} & {} & {} & {}& {} & {} & {} & {}& {} & {} & {} & {}& {} & {} & {} & {}\\
    \hline
    4 & Pengumpulan Data & {} & {} & {} & {} & \cellcolor{blue!25} & \cellcolor{blue!25} & \cellcolor{blue!25} & \cellcolor{blue!25} & \cellcolor{blue!25} & \cellcolor{blue!25} & \cellcolor{blue!25} & \cellcolor{blue!25} & {} & {} & {} & {}& {} & {} & {} & {} & {} & {} & {} & {}\\
    \hline
    5 & Analisis dan Perancangan Sistem &  {} & {} & {} & {}  & \cellcolor{blue!25} & \cellcolor{blue!25} & \cellcolor{blue!25} & \cellcolor{blue!25} & \cellcolor{blue!25} & \cellcolor{blue!25} & \cellcolor{blue!25} & \cellcolor{blue!25} & {} & {} & {} & {}& {} & {} & {} & {}& {} & {} & {} & {}\\
    \hline
    6 & Implementasi Sistem &  {} & {} & {} & {} & {} & {} & {} & {}& \cellcolor{blue!25} & \cellcolor{blue!25} & \cellcolor{blue!25} & \cellcolor{blue!25} & \cellcolor{blue!25} & \cellcolor{blue!25} & \cellcolor{blue!25} & \cellcolor{blue!25} & {} & {} & {} & {}& {} & {} & {} & {}\\
    \hline
    7 & Analisa Hasil Implementasi &  {} & {} & {} & {} & {} & {} & {} & {}& {} & {} & {} & {} & \cellcolor{blue!25} & \cellcolor{blue!25} & \cellcolor{blue!25} & \cellcolor{blue!25} & \cellcolor{blue!25} & \cellcolor{blue!25} & \cellcolor{blue!25} & \cellcolor{blue!25} & {} & {} & {} & {}\\
    \hline
    8 & Penulisan Laporan & {} & {} & {} & {} & \cellcolor{blue!25} & \cellcolor{blue!25} & \cellcolor{blue!25} & \cellcolor{blue!25}& \cellcolor{blue!25} & \cellcolor{blue!25} & \cellcolor{blue!25} & \cellcolor{blue!25}& \cellcolor{blue!25} & \cellcolor{blue!25} & \cellcolor{blue!25} & \cellcolor{blue!25}& \cellcolor{blue!25} & \cellcolor{blue!25} & \cellcolor{blue!25} & \cellcolor{blue!25}& \cellcolor{blue!25} & \cellcolor{blue!25} & \cellcolor{blue!25} & \cellcolor{blue!25}\\
    \hline
  \end{tabular}
\end{table}

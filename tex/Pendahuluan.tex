\chapter{Pendahuluan}
\section{Latar Belakang}

Riset pemrosesan bahasa alami untuk bahasa Indonesia saat ini masih terbilang sedikit.
Bahkan, masih banyak area riset yang belum tersentuh seperti contohnya
\textit{combinatory categorial grammar} (CCG).
Sementara itu, riset mengenai CCG untuk bahasa Inggris sudah cukup matang.
Adapun untuk bahasa lainnya (seperti bahasa Vietnam) sudah mulai menggunakan CCG di dalam
penelitiannya \cite{nguyen2019vietnamese}.
Agar dapat menerapkan CCG di dalam aplikasi yang dibangun, \textit{tools} seperti
CCG \textit{parser} dan CCG \textit{supertagger} harus tersedia terlebih dahulu.
Masing-masing dari \textit{tools} tersebut memerlukan \textit{dataset} agar dapat memberikan
hasil yang akurat.

Umumnya terdapat dua cara yang paling sering digunakan untuk mengembangkan CCG \textit{supertagger}
maupun CCG \textit{parser} bahasa lokal yaitu (1) membangun \textit{dataset} CCG \textit{supertag}
secara manual maupun semi-otomatis atau (2) melakukan transfer \textit{dataset} dari CCGbank
(atau dari sumber lainnya) ke dalam bahasa lokal dengan cara melakukan alih bahasa dan bila perlu
melakukan penyesuaian untuk \textit{supertag}-nya \cite{hockenmaier-steedman-2007-ccgbank}.
Proses pembangunan \textit{dataset} umumnya menggunakan bantuan \textit{annotation tool} agar
proses anotasinya menjadi lebih mudah.
Salah satu \textit{annotation tool} yang dapat digunakan adalah
CCGweb \cite{evang-etal-2019-ccgweb}.

Tugas akhir dengan judul \say{\Title} berusaha untuk membangun alat anotasi CCG baru dengan
UI/UX yang lebih baik dari CCGweb.
Selain itu, dengan bantuan NLTK alat anotasi ini dapat melakukan \textit{generate} untuk
CCG \textit{derivation}-nya kemudian pengguna juga dapat mengubah \textit{derivation}-nya
apabila diperlukan.
Tujuan dari dibangunnya alat anotasi CCG ini adalah untuk mempermudah proses anotasi yang
repetitif.
Selanjutnya, \textit{dataset} CCG pertama untuk bahasa Indonesia diharapkan dapat dipublikasikan.
\pagebreak

\section{Perumusan Masalah}
Rumusan masalah yang akan diangkat yaitu:
\begin{enumerate}
    \item Bagaimana proses pembangunan alat anotasi untuk CCG?
    \item Apakah CCGtown dapat digunakan dengan mudah?
\end{enumerate}
\section{Tujuan}
Tujuan yang diharapkan dapat tercapai oleh tugas akhir ini yaitu:
\begin{enumerate}
    \item Membangun dan merilis alat anotasi CCG semi-otomatis.
    \item Dapat digunakan untuk membangun \textit{dataset} CCG.
\end{enumerate}
\section{Rencana Kegiatan}
Rencana kegiatan yang akan dilakukan adalah sebagai berikut:
\begin{itemize}
    \item Studi literatur
    \item Studi \textit{tools} yang tersedia
    \item Studi bahasa pemrograman yang akan digunakan
    \item Perancangan sistem \textit{annotation tool} CCGtown
    \item Membangun \textit{annotation tool} CCGtown
    \item Menguji kemudahan dalam menggunakan CCGtown
\end{itemize}

% I don't know why but this table is not showing up
% \section{Jadwal Kegiatan}
% Laporan proposal ini akan dijadwalkan sesuai dengan tabel \ref{ScheduleTable}.

% \begin{table}[h!]
%   \centering
%     \caption{Jadwal kegiatan proposal tugas akhir.}
%   \label{ScheduleTable}
%   \begin{tabular}{|c|m{2.5cm}|m{0.01cm}|m{0.01cm}|m{0.01cm}|m{0.01cm}|m{0.01cm}|m{0.01cm}|m{0.01cm}|m{0.01cm}|m{0.01cm}|m{0.01cm}|m{0.01cm}|m{0.01cm}|m{0.01cm}|m{0.01cm}|m{0.01cm}|m{0.01cm}|m{0.01cm}|m{0.01cm}|m{0.01cm}|m{0.01cm}|m{0.01cm}|m{0.01cm}|m{0.01cm}|m{0.01cm}|}
%     \hline
%     \multirow{2}{*}{\textbf{No}} & \multirow{2}{*}{\textbf{Kegiatan}} & \multicolumn{24}{|c|}{\textbf{Bulan ke-}} \\
%     \hhline{~~------------------------}
%     {} & {} & \multicolumn{4}{|c|}{\textbf{1}} & \multicolumn{4}{|c|}{\textbf{2}} & \multicolumn{4}{|c|}{\textbf{3}} & \multicolumn{4}{|c|}{\textbf{4}} & \multicolumn{4}{|c|}{\textbf{5}} & \multicolumn{4}{|c|}{\textbf{6}}\\
%     \hline
%     1 & Studi Literatur & \cellcolor{blue!25} & \cellcolor{blue!25} & \cellcolor{blue!25} & \cellcolor{blue!25}& \cellcolor{blue!25} & \cellcolor{blue!25} & \cellcolor{blue!25} & \cellcolor{blue!25}& \cellcolor{blue!25} & \cellcolor{blue!25} & \cellcolor{blue!25} & \cellcolor{blue!25}& \cellcolor{blue!25} & \cellcolor{blue!25} & \cellcolor{blue!25} & \cellcolor{blue!25}& \cellcolor{blue!25} & \cellcolor{blue!25} & \cellcolor{blue!25} & \cellcolor{blue!25}& \cellcolor{blue!25} & \cellcolor{blue!25} & \cellcolor{blue!25} & \cellcolor{blue!25}\\
%     \hline
%     2 & Studi \textit{Tools} yang Tersedia & \cellcolor{blue!25} & \cellcolor{blue!25} & \cellcolor{blue!25} & \cellcolor{blue!25} &  \cellcolor{blue!25} &  \cellcolor{blue!25} &  \cellcolor{blue!25} &  \cellcolor{blue!25} & {} & {} & {} & {}& {} & {} & {} & {}& {} & {} & {} & {}& {} & {} & {} & {}\\
%     \hline
%     3 & Studi Bahasa Pemrograman & {} & {} & {} & {} & \cellcolor{blue!25} & \cellcolor{blue!25} & \cellcolor{blue!25} & \cellcolor{blue!25} & {} & {} & {} & {}& {} & {} & {} & {}& {} & {} & {} & {}& {} & {} & {} & {}\\
%     \hline
%     4 & Pengumpulan Data & {} & {} & {} & {} & \cellcolor{blue!25} & \cellcolor{blue!25} & \cellcolor{blue!25} & \cellcolor{blue!25} & \cellcolor{blue!25} & \cellcolor{blue!25} & \cellcolor{blue!25} & \cellcolor{blue!25} & {} & {} & {} & {}& {} & {} & {} & {} & {} & {} & {} & {}\\
%     \hline
%     5 & Analisis dan Perancangan Sistem &  {} & {} & {} & {}  & \cellcolor{blue!25} & \cellcolor{blue!25} & \cellcolor{blue!25} & \cellcolor{blue!25} & \cellcolor{blue!25} & \cellcolor{blue!25} & \cellcolor{blue!25} & \cellcolor{blue!25} & {} & {} & {} & {}& {} & {} & {} & {}& {} & {} & {} & {}\\
%     \hline
%     6 & Implementasi Sistem &  {} & {} & {} & {} & {} & {} & {} & {}& \cellcolor{blue!25} & \cellcolor{blue!25} & \cellcolor{blue!25} & \cellcolor{blue!25} & \cellcolor{blue!25} & \cellcolor{blue!25} & \cellcolor{blue!25} & \cellcolor{blue!25} & {} & {} & {} & {}& {} & {} & {} & {}\\
%     \hline
%     7 & Analisa Hasil Implementasi &  {} & {} & {} & {} & {} & {} & {} & {}& {} & {} & {} & {} & \cellcolor{blue!25} & \cellcolor{blue!25} & \cellcolor{blue!25} & \cellcolor{blue!25} & \cellcolor{blue!25} & \cellcolor{blue!25} & \cellcolor{blue!25} & \cellcolor{blue!25} & {} & {} & {} & {}\\
%     \hline
%     8 & Penulisan Laporan & {} & {} & {} & {} & \cellcolor{blue!25} & \cellcolor{blue!25} & \cellcolor{blue!25} & \cellcolor{blue!25}& \cellcolor{blue!25} & \cellcolor{blue!25} & \cellcolor{blue!25} & \cellcolor{blue!25}& \cellcolor{blue!25} & \cellcolor{blue!25} & \cellcolor{blue!25} & \cellcolor{blue!25}& \cellcolor{blue!25} & \cellcolor{blue!25} & \cellcolor{blue!25} & \cellcolor{blue!25}& \cellcolor{blue!25} & \cellcolor{blue!25} & \cellcolor{blue!25} & \cellcolor{blue!25}\\
%     \hline
%   \end{tabular}
% \end{table}

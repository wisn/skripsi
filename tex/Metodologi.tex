\tikzstyle{startstop} = [rectangle, rounded corners, minimum width=3cm, minimum height=1cm,text centered, draw=black, fill=red!30]
\tikzstyle{io} = [trapezium, trapezium left angle=70, trapezium right angle=110, minimum width=3cm, minimum height=1cm, text centered, draw=black, fill=blue!30]
\tikzstyle{process} = [rectangle, minimum width=3cm, minimum height=1cm, text centered, minimum width=3cm, draw=black, fill=orange!30]
\tikzstyle{decision} = [diamond, minimum width=3cm, minimum height=1cm, text centered, draw=black, fill=green!30]
\tikzstyle{arrow} = [thick,->,>=stealth]
\tikzstyle{cloud} = [draw, ellipse,fill=red!20, minimum height=2em]

\usetikzlibrary{positioning}
\usetikzlibrary{shadows}

\tikzstyle{every entity} = [top color=white, bottom color=blue!30, draw=blue!50!black!100, drop shadow]
\tikzstyle{every weak entity} = [drop shadow={shadow xshift=.7ex, shadow yshift=-.7ex}]
\tikzstyle{every attribute} = [top color=white, bottom color=yellow!20, draw=yellow, node distance=1cm, drop shadow]
\tikzstyle{every relationship} = [top color=white, bottom color=red!20, draw=red!50!black!100, drop shadow]
\tikzstyle{every isa} = [top color=white, bottom color=green!20, draw=green!50!black!100, drop shadow]


\chapter{Perancangan Sistem}

% Sebelum dapat membangun CCG \textit{supertagger}, sebuah \textit{dataset} diperlukan untuk melatih model
% \textit{classifier} MaxEnt.
% Berhubung \textit{dataset} CCG \textit{supertag} untuk bahasa Indonesia belum tersedia, tidak ada pilihan
% lain selain membuat \textit{dataset}-nya terlebih dahulu.
% Salah satu cara untuk membuat \textit{dataset} adalah dengan memanfaatkan POS \textit{tag} dari suatu
% lema yang kemudian diberikan \textit{supertag}-nya secara manual.
% Setelah \textit{dataset} memiliki \textit{supertag} yang cukup, barulah proses \textit{train} MaxEnt
% dapat dilakukan.
% Model yang telah dilatih akan disimpan ke dalam format JSON khusus agar ke depannya dapat digunakan
% tanpa perlu melakukan \textit{train} lagi.
% Setelah itu, diperlukan program yang dapat memanfaatkan model yang telah dilatih untuk membuktikan
% bahwasannya model yang telah dibangun dapat menghasilkan \textit{supertag} yang sesuai.
% Adapun langkah terakhirnya adalah dengan menyediakan dukungan \textit{web application} berbasis
% RESTful API agar siapapun dapat memanfaatkan \textit{supertagger} ini tanpa perlu terhalang oleh
% batasan bahasa pemrograman yang digunakan.

% \section{Pembuatan Dataset}

% \textit{Dataset} untuk melakukan \textit{train} CCG sayangnya belum tersedia.
% Karenanya, kita akan membuatkan \textit{dataset} secara semi otomatis dengan memanfaatkan POS
% \textit{tagger} untuk bahasa Indonesia.
% Secara formal, \textit{flowchart} untuk proses pembuatan \textit{dataset} dapat dilihat pada
% Gambar \ref{flowchart:dataset:1}.

% \begin{figure}[h!]\centering
% 	\begin{tikzpicture}[node distance=2cm]
% 		\node (start) [cloud] {Start};
% 		\node (in) [io, below of=start, yshift=0.5cm] {\textit{Input} teks bahasa Indonesia};
% 		\node (pro1) [process, below of=in, yshift=0.5cm] {POS \textit{tagging}};
% 		\node (pro2) [process, below of=pro1, yshift=0.5cm] {Jadikan \textit{queue}};
% 		\node (dec) [decision, below of=pro2, yshift=-1cm] {\textit{Queue} kosong?};
% 		\node (pro3) [process, below of=dec, yshift=-1cm] {Transformasi POS \textit{tag}};
% 		\node (out) [io, below of=pro3, yshift=0.5cm] {\textit{Output} CCG \textit{supertag}};
% 		\node (stop) [cloud, below of=out, yshift=0.5cm] {Stop};
% 		\draw [arrow] (start) -- (in);
% 		\draw [arrow] (in) -- (pro1);
% 		\draw [arrow] (pro1) -- (pro2);
% 		\draw [arrow] (pro2) -- (dec);
% 		\draw [arrow] (dec) -- node[anchor=east]{Tidak} (pro3);
% 		\draw [arrow] (pro3) -- +(-3,0) |- (dec);
% 		\draw [arrow] (dec) -- node[anchor=south]{Ya} +(4,0) |- (out);
% 		\draw [arrow] (out) -- (stop);
% 	\end{tikzpicture}
% 	\caption{Alur kerja pembuatan \textit{dataset} untuk CCG \textit{supertag}.}
% 	\label{flowchart:dataset:1}
% \end{figure}

% \subsection{Input Teks Bahasa Indonesia}

% Pada bagian \textit{input} dalam pembuatan \textit{dataset}, kita dapat mengambil teks bahasa Indonesia
% dari Indonesian Treebank\footnotemark[1]\ dan/atau dari beberapa contoh artikel yang terdapat di
% \textit{website} Wikipedia Indonesia\footnotemark[2].
% Indonesian Treebank terdapat setidaknya 1000 kalimat yang dirasa cukup untuk melakukan
% \textit{training} menggunakan model MaxEnt.
% Apa yang perlu dilakukan setelah mengambil \textit{input} adalah membersihkan bentuk \textit{tree}-nya
% untuk kemudian diambilkan kalimat yang sebenarnya.
% Caranya cukup sederhana yaitu dengan mengambil \textit{leaf} dari \textit{tree} tersebut kemudian
% disatukan di suatu pengubah dengan tipe data \textit{string}.

% Sebagai contoh, salah satu \textit{tree} yang terdapat di Indonesian Treebank dapat dilihat pada
% Gambar \ref{treebank:tree:1}.
% Apa yang dimaksud dengan \textit{leaf} pada \textit{tree} tersebut adalah $(Kera)$, $(untuk)$,
% $(*)$, $(amankan)$, $(pesta\ olahraga)$.
% Terkhusus untuk \textit{leaf} dengan bentuk spesial, seperti $(*)$, kita hilangkan sehingga teks
% yang diperoleh dari \textit{tree} tersebut adalah \say{Kera untuk amankan pesta olahraga}.
% Selanjutnya, contoh teks yang akan digunakan agar konsisten yaitu
% \say{Pamungkas dan Setyo menyukai rendang}.


% \begin{figure}\centering\small
% 	\begin{align*}
% 		&(NP\\
% 		&\ \ (NN\ (Kera))\\
% 		&\ \ (SBAR\\
% 		&\ \ \ \ (SC\ (untuk))\\
% 		&\ \ \ \ (S\ (NP-SBJ\ (*))\\
% 		&\ \ \ \ \ \ (VP\\
% 		&\ \ \ \ \ \ \ \ (VB\ (amankan))\\
% 		&\ \ \ \ \ \ \ \ (NP\ (NN\ (pesta\ olahraga)))))))
% 	\end{align*}
% 	\caption{Salah satu contoh \textit{tree} dalam Indonesian Treebank.}
% 	\label{treebank:tree:1}
% \end{figure}

% \footnotetext[1]{github.com/famrashel/idn-treebank/blob/master/Indonesian\_Treebank.bracket}
% \footnotetext[2]{id.wikipedia.org}

% \subsection{POS Tagging}

% Berdasarkan teks bahasa Indonesia yang telah diambil, kita manfaatkan \textit{tool} POS \textit{tagger}
% bahasa Indonesia untuk mendapatkan \textit{lexical category} atomik untuk masing-masing token.
% Dengan memanfaatkan POS \textit{tag} kita dapat membuat \textit{dataset} untuk CCG \textit{supertag}
% lebih mudah dibandingkan dengan memberikan \textit{tag} CCG secara manual.
% Ide dasarnya yaitu kita akan mentransformasikan POS \textit{tag} yang didapatkan menjadi CCG
% \textit{supertag} berdasarkan aturan-aturan khusus yang telah ditentukan.
% Sebagai contoh, kita dapat membuat aturan seperti mentransformasikan $\text{VB}$ menjadi
% $\text{(S$\backslash$NP)/NP}$.
% Dengan menggunakan contoh kalimat yang sama seperti di bagian sebelumnya,
% yaitu \say{Pamungkas dan Setyo menyukai rendang}, setelah menggunakan POS \textit{tagger} bahasa
% Indonesia kita dapatkan hasil sesuai dengan Gambar \ref{postag:1}.

% \begin{figure}\centering
%   \bgroup
%   \catcode`!=\active \def!{\upshape}
%   \tabbedShortstack{
%     !Pamungkas & & !dan & & !Setyo & & !menyukai & & !rendang &\\
%     \TABcline{1,3,5,7,9}
%     !{$\text{NNP}$} & &
%       !{$\text{CC}$} & &
%       !{$\text{NNP}$} & &
%       !{$\text{VB}$} & &
%       !{$\text{X}$} &
%   }
% 	\egroup
% 	\caption{Kalimat contoh dengan POS \textit{tag}-nya.}
% 	\label{postag:1}
% \end{figure}

% \subsection{Transformasi POS Tag}

% Pada bagian proses transformasi POS \textit{tag} ke bentuk CCG \textit{supertag}-nya,
% kita buatkan aturan-aturan transformasinya.
% Aturan transformasi tersebut merupakan pemetaan berbasis aturan.
% Sebagai contoh, kata \say{menyukai} memiliki POS \textit{tag} $\text{VB}$.
% Anggap saja dalam aturan transformasi terdapat pemetaan $\text{VB} \vdash \text{(S$\backslash$NP)/NP}$.
% Sehingga, kita dapatkan CCG \textit{supertag} untuk \say{menyukai} yaitu $\text{(S$\backslash$NP)/NP}$.
% Kendati demikian, masih ada bagian yang belum kita dapatkan yaitu \textit{semantic representation}-nya.
% Kita dapat menggunakan \textit{stemmer} bahasa Indonesia agar mendapatkan \textit{root words}
% dari kata \say{menyukai} yaitu \say{suka}.
% Langkah terakhirnya adalah membuatkan \textit{semantic representation}-nya berdasarkan
% \textit{root words} yang telah diperoleh sehingga kita dapatkan fungsi lambdanya yaitu
% $\lambda x.\lambda y.\ \text{suka}(y, x)$.

% Selain menggunakan \textit{stemmer}, kita dapat menggunakan \textit{morphological analyzer}.
% \textit{Morphological analyzer} salah satu kegunaannya yaitu dapat menghasilkan \textit{root words}
% sehingga dapat kita manfaatkan untuk membuat fungsi lambda.
% Untuk bahasa Indonesia, kita dapat menggunakan \textit{tool} bernama MorphInd\footnotemark[3].
% Selain MorphInd, alternatif \textit{tool} yang dapat digunakan adalah IndMA
% (Indonesian morphological analyzer).
% Namun, MorphInd dipilih karena dapat memberikan keluaran berupa morfem tersegmentasi
% \cite{larasati2011indonesian}.
% Kita dapat menggunakan MorphInd sebagai pelengkap POS \textit{tagger} untuk bahasa Indonesia.
% Hal ini agar \textit{dataset} yang dibuatkan secara semi-otomatis ini dapat memiliki kualitas yang
% baik.


% \footnotetext[3]{septinalarasati.com/morphind}

% \subsection{Output CCG Supertag}

% Keluaran dari bagian pembuatan \textit{dataset} ini adalah sebuah berkas JSON
% (JavaScript Object Notation) berisi CCG \textit{supertag} lengkap dan beberapa berkas JSON dari CCG
% \textit{supertag} yang lemanya dikelompokkan berasarkan alfabet.
% Adapun format JSON dari \textit{dataset} yang disimpan dalam berkas tersebut dapat dilihat pada Gambar
% \ref{dataset:format}.
% Untuk setiap kalimat dalam berkas tersebut direpresentasikan oleh dua objek yaitu
% (1) \say{tokens} berupa daftar token, dan
% (2) \say{supertags} berupa daftar \textit{supertag} untuk token ke-$(i, j)$ dimana $1 \leq i \leq n$
% dan $1 \leq j \leq m$ dalam sistem \textit{$1$-indexed array}.
% Sebagai contoh, kalimat \say{Pamungkas menyukai rendang} representasinya dapat dilihat pada Gambar
% \ref{dataset:example}.
% Demikian itu, kita dapat mengambil \textit{dataset} secara lengkap dengan CCG \textit{supertag}-nya.

% \begin{figure}\centering
% 	\bgroup
% 	\catcode`!=\active \def!{\upshape}
% 	\tabbedShortstack[l]{
% 		!{[} & & &\\
% 			& !{\{} & &\\
% 			& & !{\say{tokens}:} &
% 				!{[\say{$token_{1,1}$}, \say{$token_{1,2}$}, $\dots$, \say{$token_{1,m}$}],}\\
% 			& & !{\say{supertags}:} &
% 				!{[\say{$supertag_{1,1}$}, \say{$supertag_{1,2}$}, $\dots$, \say{$supertag_{1,m}$}]}\\
% 			& !{\}}, & &\\
% 			& !{\{} & &\\
% 			& & !{\say{tokens}:} &
% 				!{[\say{$token_{2,1}$}, \say{$token_{2,2}$}, $\dots$, \say{$token_{2,m}$}],}\\
% 			& & !{\say{supertags}:} &
% 				!{[\say{$supertag_{2,1}$}, \say{$supertag_{2,2}$}, $\dots$, \say{$supertag_{2,m}$}]}\\
% 			& !{\}}, & &\\
% 			& & &\\
% 			& & &\\
% 			& $\vdots$ & &\\
% 			& & &\\
% 			& & &\\
% 			& !{\{} & &\\
% 			& & !{\say{tokens}:} &
% 				!{[\say{$token_{n,1}$}, \say{$token_{n,2}$}, $\dots$, \say{$token_{n,m}$}],}\\
% 			& & !{\say{supertags}:} &
% 				!{[\say{$supertag_{n,1}$}, \say{$supertag_{n,2}$}, $\dots$, \say{$supertag_{n,m}$}]}\\
% 			& !{\}} & &\\
% 		!{]} & & &
% 	}
% 	\egroup
% 	\caption{Format JSON \textit{dataset} yang disimpan di dalam berkas.}
% 	\label{dataset:format}
% \end{figure}

% % \begin{figure}\centering
% % 	\bgroup
% % 	\catcode`!=\active \def!{\upshape}
% % 	\catcode`?=\active \def?#1{\makebox[0pt]{#1}}
% % 	\tabbedShortstack[l]{
% % 		!{[} & & &\\
% % 			& !{\{} & &\\
% % 			& & !{\say{tokens}: [} &\\
% % 			& & !{\ \ \ \ \say{Pamungkas},} &\\
% % 			& & !{\ \ \ \ \say{dan},} &\\
% % 			& & !{\ \ \ \ \say{Setyo},} &\\
% % 			& & !{\ \ \ \ \say{menyukai},} &\\
% % 			& & !{\ \ \ \ \say{rendang},} &\\
% % 			& & !{\ \ \ \ \say{.}} &\\
% % 			& & !{],} &\\
% % 			& & !{\say{supertags}: [} &\\
% % 			& & !{\ \ \ \ \say{$\text{NP: \so{pamungkas}}$},} &\\
% % 			& & !{\ \ \ \ \say{$\text{CONJ: $\lambda x.\lambda y.\lambda f.\ (f\ x) \land (f\ y)$}$},} &\\
% % 			& & !{\ \ \ \ \say{$\text{NP: \so{setyo}}$},} &\\
% % 			& & !{\ \ \ \ \say{$\text{(S$\backslash$NP)/NP: $\lambda x.\lambda y.\ suka(y, x)$}$},} &\\
% % 			& & !{\ \ \ \ \say{$\text{NP: \so{rendang}}$},} &\\
% % 			& & !{\ \ \ \ \say{$\text{Z}$}} &\\
% % 			& & !{]} &\\
% % 			& !{\}} & &\\
% % 		!{]} & & &
% % 	}
% % 	\egroup
% % 	\caption{Contoh isi dari berkas \textit{dataset} dalam format JSON.}
% % 	\label{dataset:example}
% % \end{figure}

% \begin{figure}\centering\scriptsize
% 	\bgroup
% 	\catcode`!=\active \def!{\upshape}
% 	\tabbedShortstack[l]{
% 		!{[} & & &\\
% 			& !{\{} & &\\
% 			& & !{\say{tokens}:} &
% 				!{[\say{Pamungkas}, \say{menyukai}, \say{rendang}, \say{.}],}\\
% 			& & !{\say{supertags}:} &
% 				!{[
% 					\say{$\text{NP: \so{pamungkas}}$},
% 					\say{$\text{(S$\backslash$NP)/NP: $\lambda x.\lambda y.\ suka(y, x)$}$},
% 					\say{$\text{NP: \so{rendang}}$},
% 					\say{$\text{Z}$}
% 				]}\\
% 			& !{\}} & &\\
% 		!{]} & & &
% 	}
% 	\egroup
% 	\caption{Contoh isi dari berkas \textit{dataset} dalam format JSON.}
% 	\label{dataset:example}
% \end{figure}

% \section{Mempersiapkan CCG Supertagger}

% \subsection{Melatih Model MaxEnt}

% Model yang akan digunakan oleh \textit{supertagger} ini adalah Maximum Entropy (MaxEnt).
% Model ini dipilih karena ketersediaan \textit{dataset} bahasa Indonesia yang masih sangat kurang.
% Selain itu, model MaxEnt digunakan di sebuah riset yang dilakukan oleh Stephen Clark dalam
% pembuatan \textit{supertagger}-nya.
% Bahkan, performansi \textit{supertagger} yang dikembangkan sangat baik.
% Riset tersebut pada intinya membuktikan bahwasannya MaxEnt dapat digunakan di \textit{supertagger}
% juga meskipun pada peruntukannya MaxEnt dibuat untuk POS \textit{tagger}.
% Tentunya terdapat beberapa penyesuaian yang harus dilakukan.
% Salah satunya adalah formula probabiltas yang digunakan.
% Formula yang telah disesuaikan tersebut dapat dilihat di persamaan \ref{maxent:equation:2}.

% Model yang telah dilatih akan disimpan di dalam sebuah berkas sebagai \textit{snapshot} agar apabila
% proses \textit{train} mengalami kegagalan kita tidak perlu mengulangi proses \textit{train} dari
% awal lagi.
% Dalam hal ini, untuk setiap $10$ \textit{dataset} yang telah diproses, proses penyimpanan
% \textit{snapshot} akan dilakukan.
% Adapun format \textit{snapshot} menggunakan JSON yang menyimpan informasi berupa jumlah kalimat
% terproses dan sebuah tabel yang menyimpan kalkulasi pada saat \textit{snapshot} disimpan.
% Setelah proses \textit{train} selesai, \textit{snapshot} akan diperbarui kemudian digunakan
% untuk membuat sebuah berkas baru berupa JSON yang menyimpan \textit{trained model} tersebut
% sehingga \textit{supertagger} yang dibangun dapat memuat model tersebut tanpa perlu melakukan
% \textit{train} kembali.

% \begin{figure}\centering
% 	\bgroup
% 	\catcode`!=\active \def!{\upshape}
% 	\tabbedShortstack[l]{
% 		!{$\{$} &\\
% 			& !{\say{starting\_point}: $70$,}\\
% 			& !{\say{rows}:
% 				\{\say{$token_1$}: $1$, \say{$token_2$}: $2$, $\dots$, \say{$token_n$}: $N$\},}\\
% 			& !{\say{cols}:
% 				 \{\say{$token_1$}: $1$, \say{$token_2$}: $2$, $\dots$, \say{$token_m$}: $M$\},}\\
% 			& !{\say{values}: [}\\
% 			& !{\ \ \ \ [$value_{1,1}$, $value_{1,2}$, $\dots$, $value_{1,m}$],}\\
% 			& !{\ \ \ \ [$value_{2,1}$, $value_{2,2}$, $\dots$, $value_{2,m}$],}\\
% 			& \\
% 			& \\
% 			& \ \ \ \ $\vdots$\\
% 			& \\
% 			& \\
% 			& !{\ \ \ \ [$value_{n,1}$, $value_{n,2}$, $\dots$, $value_{n,m}$]}\\
% 			& !{]}\\
% 		!{$\}$} &
% 	}
% 	\egroup
% 	\caption{Format JSON untuk menyimpan \textit{snapshot}.}
% 	\label{snapshot:format}
% \end{figure}

% Pada Gambar \ref{snapshot:format}, \textit{field} \say{$starting\_point$} menyimpan informasi berupa
% bilangan cacah untuk memberikan tanda pada elemen ke-$i$ sebaiknya proses \textit{train} dimulai.
% Dalam hal ini, berdasarkan contoh, sebaiknya dimulai dari elemen ke-$70$.
% Selanjutnya, \textit{field} \say{$rows$} dan \say{$cols$} masing-masing menyatakan daftar baris dan
% kolom beserta nomor \textit{index}-nya untuk digunakan oleh \textit{field} \say{$values$} nantinya.
% \textit{Field} \say{$values$} merupakan representasi dari tabel yang memiliki $N$ baris serta $M$
% kolom.
% Adapun \say{$value$} pada elemen ke-$(i, j)$ menyimpan informasi berupa objek untuk model MaxEnt
% setelah proses \textit{train} pada \say{$starting\_point$} saat itu.
% Objek yang dimaksud berupa \say{konteks} yaitu berisikan daftar $k$ token sebelum dan sesudah token
% ke-$(i, j)$ juga daftar $l$ \textit{supertag} sebelum dan sesudah \textit{supertag} ke-$(i, j)$.
% Bahkan, \say{$value$} dapat pula meyimpan \textit{prefix} dan \textit{suffix} dari token ke-
% $(i, j)$ tersebut.

% \subsection{Membangun Library Haskell}

% Implementasi \textit{supertagger} bahasa Indonesia ini akan ditulis dalam bahasa pemrograman Haskell.
% Haskell dipilih karena kemampuannya dalam mengolah data teks terutama kegiatan \textit{parsing} berkat
% desain bahasa serta dukungan \textit{library} yang telah tersedia.
% Secara desain bahasa, Haskell sangat dekat dengan \textit{category theory} sehingga diharapkan
% \textit{supertag} yang disimpan di dalam program nantinya dapat berupa \textit{algebraic data type}
% yang semirip mungkin dengan \textit{category theory}.
% Hal ini menjadi pertimbangan karena tujuan akhir dari riset-riset yang berhubungan dengan CCG adalah
% untuk membangun CCG \textit{parser}.

% Pembangunan \textit{library} ini sangat penting karena nantinya \textit{library} ini dapat dimanfaatkan
% untuk membangun program \textit{command line interface} (CLI), \textit{web application}, dan sebagainya.
% Dalam hal ini, \textit{supertagger} yang dibangun dalam tugas akhir ini akan memiliki dukungan RESTful
% API agar tidak ada batasan dalam bahasa pemrograman yang harus digunakan untuk memanfaatkan
% \textit{supertagger} ini.
% Sangat mungkin pula nantinya terdapat program yang memanfaatkan \textit{library} Haskell ini secara
% langsung di programnya.
% Demikian itu, menyediakan \textit{library} untuk suatu bahasa pemrograman (dalam hal ini, Haskell)
% dirasa perlu.

% \subsection{Menguji Model MaxEnt}

% Pengujian model MaxEnt setelah proses \textit{train} dilakukan dengan cara mencocokkan \textit{supertag}
% yang dihasilkan oleh \textit{supertagger} dengan \textit{supertag} yang telah diberikan secara manual
% terhadap suatu kalimat yang diberikan.
% Dengan demikian, setidaknya akan ada dua \textit{dataset} yaitu \textit{data train} dan
% \textit{data test}.
% \textit{Data train} dibuat secara semi-otomatis sesuai dengan bagian \say{Pembuatan Dataset}.
% Selanjutnya, \textit{data test} dibuat secara manual yang mana berjumlah $20\%$ hingga $30\%$ dari
% jumlah kalimat pada \textit{data train}.
% Sebagai catatan, untuk kata atau token yang tidak ada di \textit{data train} tetapi ada di
% \textit{data test} (\textit{unseen words}) akan diberikan \textit{supertag} berupa
% \textit{category} $\text{N}$.

% \section{Membangun CCG Supertagger Versi CLI}

% Sebagai langkah awal, program CLI akan dibangun lebih dahulu sebelum versi \textit{web application}-nya
% rilis.
% Hal ini agar validasi dapat dilakukan yaitu untuk memeriksa apakah keluaran dari \textit{supertagger}
% sudah sesuai.
% Adapun \textit{flowchart} dari program CLI \textit{supertagger} ini terdapat pada Gambar
% \ref{flowchart:supertagger:1}.

% \begin{figure}[h!]\centering
% 	\begin{tikzpicture}[node distance=2cm]
% 		\node (start) [cloud] {Start};
% 		\node (in) [io, below of=start, yshift=0.5cm] {\textit{Input} teks bahasa Indonesia};
% 		\node (pro1) [process, below of=in, yshift=0.5cm] {\textit{Tokenize}};
% 		\node (pro2) [process, below of=pro1, yshift=0.5cm] {Jadikan \textit{queue}};
% 		\node (dec) [decision, below of=pro2, yshift=-1cm] {\textit{Queue} kosong?};
% 		\node (pro3) [process, below of=dec, yshift=-1cm] {Menghitung probabilitas terhadap token};
% 		\node (pro4) [process, below of=pro3, yshift=0.5cm] {Memilih \textit{supertag} yang sesuai};
% 		\node (out) [io, below of=pro4, yshift=0.5cm] {\textit{Output} CCG \textit{supertag}};
% 		\node (stop) [cloud, below of=out, yshift=0.5cm] {Stop};
% 		\draw [arrow] (start) -- (in);
% 		\draw [arrow] (in) -- (pro1);
% 		\draw [arrow] (pro1) -- (pro2);
% 		\draw [arrow] (pro2) -- (dec);
% 		\draw [arrow] (dec) -- node[anchor=east]{Tidak} (pro3);
% 		\draw [arrow] (pro3) -- (pro4);
% 		\draw [arrow] (pro4) -- +(-4,0) |- (dec);
% 		\draw [arrow] (dec) -- node[anchor=south]{Ya} +(4,0) |- (out);
% 		\draw [arrow] (out) -- (stop);
% 	\end{tikzpicture}
% 	\caption{Alur kerja proses \textit{tagging}.}
% 	\label{flowchart:supertagger:1}
% \end{figure}

% \subsection{Tokenize}

% Pertama-tama, \textit{toknize} akan memisahkan antar kalimat yang dipisahkan oleh karakter titik
% (\say{.} tanpa tanda kutip) yang diikuti oleh spasi. Selanjutnya, untuk setiap kalimat,
% \textit{tokenize} akan memisahkan beberapa bagian yang terindikasi sebagai token.
% Token umumnya berupa kata yang dipisahkan oleh spasi.
% Namun, token juga dapat berupa karakter khusus seperti karakter tanda koma (\say{,} tanpa tanda kutip).
% Dengan demikian, \textit{supertagger} dapat dengan mudah melakukan kalkulasi probabilitas untuk
% masing-masing token.
% Proses \textit{tokenize} pada awalnya terlihat mudah.
% Sayangnya, proses ini cukup \textit{tricky} karena ada beberapa hal tidak umum yang seringkali muncul.
% Sebagai contoh, \say{Rp5.000,00} memiliki dua token yaitu \say{Rp} sebagai \textit{currency} dan
% \say{5.000,00} sebagai nominalnya.

% \subsection{Menghitung Probabilitas}

% Probabilitas untuk sebuah token ke-$(i, j)$ dapat dihitung dengan menggunakan \textit{beam search}.
% Persamaan yang digunakan adalah persamaan \ref{maxent:equation:4}.
% Persamaan tersebut memiliki sebuah \textit{category sequence} $C$ serta kalimat $S$ sebagai syarat
% dari probabilitas tersebut.

% \begin{equation}\label{maxent:equation:4}
% 	p(C|S) = \prod_i p(c_i, h_i)
% \end{equation}

% \noindent
% dimana $c_i$ adalah \textit{category} ke-$i$ dalam \textit{sequence} tersebut dan $h_i$ berupa konteks
% dari token ke-$i$.
% \textit{Beam search} digunakan untuk mendapatkan $N$ \textit{sequence} teratas saat proses
% \textit{tagging} dilakukan.
% Dalamn hal ini, $N = 10$ digunakan.

% \subsection{Memilih Supertag}

% Setelah proses penghitungan probabilitas selesai dilakukan, proses pemilihan \textit{supertag} dilakukan
% berdasarkan $N$ \textit{sequence} teratas yang disimpan.
% Berhubung MaxEnt yang digunakan adalah \textit{conditional maximum entorpy}, maka selanjutnya akan
% dilakukan proses pemilihan \textit{supertag} berdasarkan kondisi-kondisi yang terpenuhi.

CCGtown dibangun dengan menggunakan bahasa pemrograman Python dan JavaScript.
Adapun \textit{framework} yang digunakan adalah Django.
Versi awal CCGtown merupakan sebuah \textit{proof-of-concept} dari
\textit{open source graphical annotation tool} berbasis web yang dilengkapi dengan fitur
penganotasian semi-otomatis.
Bahasa pemrograman Python digunakan karena sebagian besar \textit{library} untuk CCG
sudah tersedia di PyPi \footnote{\url{https://pypi.org/}}.
Salah satu \textit{library} penting yang digunakan sebagai dasar dari fitur penganotasian
semi-otomatis adalah NTLK \footnote{\url{http://www.nltk.org/}}.
Selanjutnya, Django digunakan untuk mempercepat proses pengembangan aplikasi.
Adapun JavaScript digunakan untuk menjadikan CCGtown aplikasi berbasis web yang interaktif.

Alur kerja CCGtown pada umumnya adalah (1) pengguna melakukan registrasi, (2) pengguna
melakukan \textit{login} ke sistem, (3) pengguna membuat proyek baru, (4) pengguna
menambahkan kalimat yang ingin dianotasi, (5) pengguna melakukan anotasi kemudian melakukan
\textit{generate} CCG \textit{derivation} dan/atau melakukan modifikasi
\textit{derivation}-nya apabila diperlukan, dan (6) pengguna melakukan \textit{export}
setelah selesai melakukan anotasi. Alur kerja tersebut mempengaruhi desain sistem dari
CCGtown.

\section{Desain Database}
CCGtown menggunakan PostgreSQL sebagai
DBMS\footnote{Database Management System}-nya.
Hal ini karena PostgreSQL memiliki kemampuan untuk menyimpan struktur data
JSON\footnote{JavaScript Object Notation} sehingga memudahkan CCGtown untuk menyimpan
format JSON dari CCG \textit{derivation} yang telah dimanipulasi oleh pengguna melalui
fitur \textit{editable CCG derivation}.
PostgreSQL juga memiliki banyak fitur lain termasuk di antaranya dukungan
dari \textit{non-relational database model} (seperti \textit{multi-model graph})
sehingga apabila di waktu yang akan datang CCGtown memerlukan perubahan signifkan
terhadap desain \textit{database}-nya tidak perlu mengganti DBMS yang digunakan.
Fitur lain seperti \textit{function} dan \textit{procedure} juga akan sangat membantu
pengembangan CCGtown di waktu yang akan datang.

CCGtown versi awal sejatinya hanya membutuhkan tiga tabel saja yaitu tabel
\textit{accounts} untuk menyimpan pengguna yang terdaftar, tabel
\textit{projects} untuk menyimpan proyek-proyek yang sudah dibuat, dan tabel
\textit{sentences} untuk menyimpan kalimat-kalimat yang akan dianotasikan.
Tiga tabel tersebut sudah cukup untuk membangun \textit{proof-of-concept} dari
alat anotasi CCG yang akan dibangun. Adapun
ERD\footnote{Entity Relationship Diagram}-nya dapat dilihat pada Gambar\ref{erd-1}.

\begin{figure}\centering\small
  \scalebox{.87}{
  \begin{tikzpicture}[node distance=1.5cm, every edge/.style={link}]
    \node[entity] (acc) {Accounts};
    \node[attribute] (acc-id) [above=of acc] {\key{ID}} edge (acc);
    \node[attribute] (acc-uuid) [above right=of acc] {\key{UUID}} edge (acc);
    \node[attribute] (acc-email) [right=of acc] {Email} edge (acc);
    \node[attribute] (acc-password) [below right=of acc] {Password} edge (acc);

    \node[relationship] (creates) [left=of acc] {Creates} edge (acc);

    \node[entity] (prj) [below=of creates] {Projects} edge (creates);
    \node[attribute] (prj-id) [above right=of prj] {\key{ID}} edge (prj);
    \node[attribute] (prj-uuid) [above left=of prj] {\key{UUID}} edge (prj);
    \node[attribute] (prj-author) [left=of prj] {Author ID} edge (prj);
    \node[attribute] (prj-name) [below left=of prj] {Name} edge (prj);
    \node[attribute] (prj-status) [below=of prj] {Status} edge (prj);
    \node[attribute] (prj-rules) [below right=of prj] {Rules} edge (prj);

    \node[relationship] (adds) [right=0.5cm and 2cm of prj] {Adds} edge (prj);

    \node[entity] (snt) [below=of adds] {Sentences} edge (adds);
    \node[attribute] (snt-id) [left=of snt] {\key{ID}} edge (snt);
    \node[attribute] (snt-uuid) [below left=of snt] {\key{UUID}} edge (snt);
    \node[attribute] (snt-project) [below=of snt] {Project ID} edge (snt);
    \node[attribute] (snt-words) [below right=of snt] {Words} edge (snt);
    \node[attribute] (snt-cats) [right=of snt] {Categories} edge (snt);
    \node[attribute] (snt-deriv) [above right=of snt] {Derivations} edge (snt);
  \end{tikzpicture}
  }
  \caption{Conceptual Entity Relationship Diagram (ERD) CCGtown}
  \label{erd-1}
\end{figure}

Masing-masing tabel memiliki dua \textit{key} yaitu $ID$ dan
$UUID$\footnote{Universally Unique IDentifier}.
$ID$ merupakan \textit{primary key} \textit{integer} dengan \textit{auto increment}
yang berfungsi sebagai \textit{identifier} untuk melakukan operasi
\textit{update} maupun \textit{delete}.
Adapun $UUID$ merupakan \textit{indexed column} yang berfungsi sebagai
\textit{indentifier} publik (dapat dilihat oleh pengguna melalui URL)
yang mana digunakan untuk operasi \textit{read}.
$ID$ tidak digunakan sebagai \textit{identifier} publik karena pengguna dapat
melakukan \textit{brute-force} untuk mencari proyek ataupun kalimat berdasarkan
$ID$ yang bukan miliknya.
Demikian itu alasan ditambahkannya atribut $UUID$.
Alasan kenapa CCGtown tetap menyimpan kolom $ID$ adalah karena $ID$ nantinya akan
digunakan untuk membuat \textit{pagination}.

Pada tabel $accounts$, selain $ID$ dan $UUID$ juga memiliki atribut $email$ dan
$password$. Masing-masing atribut tersebut menggunakan tipe data \textit{string}
atau VARCHAR di PostgreSQL.
Tabel $accounts$ memiliki hubungan \textit{one-to-many} terhadap tabel $projects$.
Adapun atribut tabel $projects$ adalah $author\_id$, $name$, $status$, dan $rules$.
Atribut $author\_id$ merupakan \textit{foreign key} (\textit{indexed}) yang
mengarah kepada tabel $accounts$ dan tipe data yang digunakan sama dengan
atribut $ID$ yang terdapat di tabel $accounts$.
Atribut $name$ menggunakan tipe data \textit{string} (VARCHAR).
Atribut $status$ menggunakan tipe data \textit{integer} yang berperan sebagai
\textit{enum} ($0$ = \textit{just created}, $1$ = \textit{in progress},
$2$ = \textit{finished}, dan $3$ = \textit{dropped}).
Tabel $projects$ memiliki hubungan \textit{one-to-many} terhadap tabel $sentences$.
Adapun atribut tabel $sentences$ adalah $project\_id$, $words$, $categories$, dan
$derivations$. Atribut $project\_id$ merupakan \textit{foreign key}
(\textit{indexed}) yang mengarah kepada tabel $projects$ dan tipe data yang digunakan
sama dengan atribut $ID$ yang terdapat di tabel $projects$.
Sisanya, atribut $words$, $categories$, dan $derivations$ menggunakan tipe data JSON.


\section{Desain Sistem}
TBA.


\section{Pertimbangan UI/UX}
TBA.
